%! Author = kulagins
%! Date = 02.05.24

% Preamble
\documentclass[11pt]{article}

% Packages
\usepackage{amsmath}

% Document
\begin{document}

    \subsection{New heuristic: Local search}
    One strategy that can be used is local search, a similar approach as with swaps.
    The starting point could be a previously valid solution (for example, one by HetPart).
    In the first step, the solution would need to become valid again.
    Then, we repartition blocks on the critical path and merge blocks in attempt to improve makespan.
    Finally, we swap blocks between processors to finish makespan improvement.

    \subsubsection{Restore Validity}
    We~(Algorithm~\ref{alg:RestoreValidity}) start with the previously partitioned and assigned quotient tree $\Gamma$,
    and the new weights on tasks and edges.
    We first recompute the memory requirements of all blocks.
    Then we identify all blocks that have become invalidated and exceed the memory of their assigned processor
    (Line~\ref{line:damaged}).
    For all these blocks, we try out ``border'' tasks, that is, tasks, that have neighbours in neighbouring blocks
    (Line~\ref{line:neighb}).
    We tentatively move them (creating a temporary quotient DAG $\Gamma'$) and assess the impact of this move on the
    makespan(Line~\ref{line:makespan}).
    The condition is, of course, that the neighboring block now does not exceed the memory of its own processor.
    The task that delivers the best makespan in then moved in Line~\ref{line:movebest}.
    The entire process is being repeated until the block fits into the memory of the processor again.
    We return the modified quotient graph $\Gamma$ that contains the mapping to processors.

    \begin{algorithm}[h!]
        \caption{Restore Validity}
        \label{alg:RestoreValidity}
        \begin{algorithmic}[1]
            \Procedure{Restore Validity}{$\Gamma$, $W$ }
                \\
                \Comment{Input: initial schedule as quotient tree $\Gamma$, new weights $W$}
                \State $\Gamma \gets $ \textsc{QuotientDAG}($\Gamma$, $W$ ); \Comment Quotient DAG with new weights

                \State $D \gets \emptyset$ \Comment {Damaged blocks that require action}
                \For{$\nu_i \in \Gamma$}
                    \State $r_{\nu_m} \gets W$
                    \Comment{Recompute the new memory requirements of blocks in the quotient DAG}
                \EndFor
                \For{$\nu_i \in \Gamma$}
                    \label{line:damaged}
                    \If{$r_{\nu_i} > M_{proc(\nu_i)}$}
                        \State $D \gets D \cup \{\nu_i\}$
                    \EndIf
                \EndFor

                \For{$\nu_i \in D$}
                    \Comment Restore the validity of each damaged block
                    \While{$r_{\nu_i} > M_{proc(\nu_i)}$}
                        \State $MS_{min}\gets \infty; v_{min} \gets NULL; \nu_{min} \gets NULL$
                        \Comment Moving $v_{min}$ to $\nu_{min}$ gives $MS_{min}$
                        \For{$v \in \nu_i$}
                            \For{$u \in ( \Pi_v \cup C_v) \cap (\Pi_{\nu_i} \cup C_{\nu_i})$}
                                \label{line:neighb}
                                \State $\Gamma' \gets \Gamma \mid \nu(v) == \nu(u)$
                                \If{ $r_{\nu(u)} > M_{proc(\nu(u))}$ and \textsc{Makespan}($\Gamma'$)$\leq MS_{min}$}
                                    \label{line:makespan}
                                    \State $MS_{min} \gets $\textsc{Makespan}($\Gamma'$); $v_{min} \gets v$; $\nu_{min} \gets \nu(u);$
                                \EndIf
                            \EndFor
                        \EndFor
                        \State  $\Gamma \gets \Gamma \mid \nu(v_{min}) == \nu_{min}$ \label{line:movebest}
                    \EndWhile
                \EndFor
                \State return $\Gamma$
            \EndProcedure
        \end{algorithmic}
    \end{algorithm}

    \subsubsection{Repartition and Merge}

    In the next step, we center our attention around changes in the execution environments and how the changes in
    the workflow reflect on it.

    First, we try to further partition blocks that lie on the critical path of the workflow.
    We first compute the critical path (Line~\ref{line:cp}) and then, as long as there are free processors in the
    execution environment(Line~\ref{line:whilefree}, assumed to be sorted in descending order of memory sizes),
    we partition the blocks on the critical path(Line~\ref{line:repartition}).
    We try to partition each block into 2 smaller blocks.
    However, due to the nature of the partitioner, more can be created.
    We assume that the resulting blocks(Line~\ref{line:repartition}) are given back sorted by their memory requirement.
    We then assign them to free processors as long as the blocks fit and there are free processors.
    If even one part does not fit(Line~\ref{line:ifnotfits}), then we discard the entire partition and try partitioning
    the next block on the critical path.

    Not only partitioning (introducing more parallelism) can lead to a makespan improvement.
    Shorter runtimes of tasks can justify merging some blocks together.
    For all pairs of blocks that are neighbours (Line~\ref{Line:neighbblocks}), we tentatively merge them and assign
    the result to the processor of the first block.
    If the merged block fits on this processor and the merged quotient graph has a smaller overall makespan, then we save
    the merge and proceed with other blocks.

    \begin{algorithm}[h!]
        \caption{Repartition and Merge}
        \label{alg:RepartitionMerge}
        \begin{algorithmic}[1]
            \Procedure{Repartition and Merge}{$\Gamma$, $\cluster$ }
                \\
                \Comment{Input: quotient tree $\Gamma$, execution environment $\cluster$}
                \State $CP \gets $\textsc{Criticalpath}($\Gamma$) \label{line:cp}
                \State $Free \gets \{ p \in \cluster \mid p$ is free$\}$
                \While{$\mid Free \mid \neq 0$}
                    \label{line:whilefree}
                    \For{$\nu \in CP$}
                        \State $\{ \nu_1, \nu_2,\dots\}\gets$\textsc{Partition}($\nu$,2) \label{line:repartition}
                        \For{$\nu' \in \{ \nu_1, \nu_2,\dots\}$}
                            \If{$r_\nu' \leq Free.$head()}
                                \State proc($\nu'$)$\gets Free.$head();
                            \Else
                                \label{line:ifnotfits}
                                \State \textsc{Merge}$\nu' \in \{ \nu_1, \nu_2,\dots\}$; break;
                                \Comment Merge the parts back and leave this partitioning, as it does not fit
                            \EndIf
                        \EndFor
                    \EndFor
                \EndWhile

                \For{$\nu_1, \nu_2 \in \Gamma \mid \nu_1 \in ( \Pi_{\nu_2} \cup C_{\nu_2})$}
                    \label{line:neighbblocks}
                    \State $\{\Gamma', \nu_m\} \gets $ \textsc{Merge}($\nu_1, \nu_2$); proc($\nu_m$)$\gets $ proc($\nu_1$);
                    \State $\mu \gets $\textsc{Makespan}($\Gamma'$)
                    \If{$r_{\nu_m} \leq M_{proc(\nu_m)}$ and $\mu \leq $ \textsc{Makespan}($\Gamma$)}
                        \State $\Gamma \gets \Gamma'$
                    \EndIf
                \EndFor

                \State return $\Gamma$
            \EndProcedure
        \end{algorithmic}
    \end{algorithm}

    \subsubsection{Swaps}

    Then a process similar to Step 4 in %\daghetpart
    can be executed: the newly rearranged blocks are swapped until no
    improvement in makespan can be achieved.

\end{document}
%! Author = kulagins
%! Date = 02.05.24

% Preamble
\documentclass[11pt]{article}

% Packages
\usepackage{amsmath}

% Document
\begin{document}

    \paragraph{Workflow-related changes}

    \begin{itemize}
        \item A task $v$ takes longer or shorter to execute than planned: its time weight $w_u$ changes to $w'_u$.
        \item A task $v$ takes more or less memory to execute than planned: its memory requirement $m_v$ changes to $m'_v$.

    \end{itemize}

    The following changes are not a part of this article's scope:

    \begin{itemize}
        \item The workflow structure changes: edges or tasks come in or leave.
    \end{itemize}

    \paragraph{Execution environment-related changes }


    \begin{itemize}
        \item A processor exists the execution environment: $k$ decreases and $\cluster$ changes.
        \item A processor enters the execution environment: $k$ increases, $\cluster$ gets a new processor with possibly new memory requirement and processor speed.

    \end{itemize}

    The following changes are not a part of this article's scope:

    \begin{itemize}
        \item Processor characteristics change: the memory requirement or speed become bigger or smaller
    \end{itemize}

    \subsection{Time of changes }

    We consider discrete time in seconds.
    The time point(s) at which the changes happen is unambiguously defined.

    For any task $v$, its runtime equals its time weight divided by the speed of the processor $p_j$ it has been assigned to: $w_v/s_j$.
    The start time of any task $v$ is its top level($\bar{l}_v$), or the difference between the maximum bottom level in the workflow (the makespan of the workflow) and the task's own bottom level: $\bar{l}_v = \mu_\Gamma - \bottomlevel{v}$.
    The start time of the source task in the workflow is zero.
    The end time of a task $v$ is its start time and its runtime: $\bar{l}_v + w_v/s_j$

    \subsection{Changes and knowledge horizon - important questions TBA}

    Given a valid mapping of tasks to processors, we can say what we predicted would happen at any given time point $T$: what tasks have been executed, what have not finished or have not even started.

    At the point of change, we know that some tasks that finished took longer than expected ($w_v$ are bigger) or shorter.
    However, how do we model the following:
    \begin{itemize}
        \item Do we know the new weights of currently running tasks and tasks that have not yet started? This means, do we foresee into the future or do we assume that all weights on unfinished tasks remain the same?
        \item A change in memory requirements can mean that the assignment had been invalid. Do we assume that these tasks failed and we need to rerun them?
        \item How many times of change do we model - one per workflow run, or multiple?
        \item At what time does the change and reevaluation happen - is it a fixed (random?) point of time or is it workflow-dependent (say, after 10\% of the workflow is ready)?
    \end{itemize}


    \subsection{Dynamic Scenario}

    In a workflow execution environment, the scheduling method interacts with the runtime environment, which provides information such as resource estimates.
    This information may include, memory usage, runtime, graph structures, or the status of the underlying infrastructure.
    In order to ensure that the information is up to date, a monitoring system observes the workflow execution and collects metrics for tasks and the underlying infrastructure.
    By incorporating dynamic monitoring values, e.g., the resources a task consumed, the runtime environment can incorporate the data into the prediction model to provide more accurate resource predictions.
    Also the underlying infrastructure can change during the workflow execution.
    Examples are processor failures, node recoveries, or acquisition of new nodes.
    However, also when the hardware of the infrastructure does not change, the set of nodes provided as a scheduling target might change due to release or occupation in shared cluster infrastructures.
    As infrastructure information and resource predictions are dynamically updated and provided to the scheduler during the workflow runtime, the previous schedule becomes invalid and a new one must be calculated.

    For state-of-the-art memory prediction methods, a cold-start median prediction error for heterogeneous infrastructures of approximately 15\% is shown~\cite{}.
    Online prediction methods were able to significantly reduce the error during runtime, with the reduction reaching up to one-third of the cold-start error~\cite{baderDiedrichDynamic2023,witt2019learning}.
%For instance, Nadeen~et~al.\cite{} report an error of 10\%, 11\%, and 15\% while the task prediction errors shows a normal and exponential distribution.
%Bader~et~al.~ report a prediction error between 13\% and 17\% for their method, showing an exponential task error distribution.
% @Svetlana, willst du sowas für deine Experimente? Also die Daten, welche du dann konfigurieren kannst?
    Such a dynamic execution environment necessitates for a dynamic scheduling method where the schedule can be recomputed during the workflow execution.

    \subsection{Retracing the effects of change on an existing schedule}
    After the monitoring system has reported changes, we need to assess their impact on the existing schedule.
    These changes can invalidate the schedule (\eg if there is not enough memory for some tasks to execute anymore),
    they can lead to a later finishing time (\eg if some tasks longer and delay other tasks), or they can have no effect (\eg if new processors
    joined the cluster, but the old schedule did not account for them).
    To assess the impact, we need to retrace the schedule.

    First, we find out if at least one processor that had assigned tasks has exited - this instantly invalidates the
    entire schedule.

    We then iterate over all tasks of the workflow in a topological order - any of the orderings given by rankings BL, BLC or MM
    is a topologial ordering.
    We then repeat steps similar to those we did during tentative assignment in our heuristic, except we do not choose a processor
    anymore, but rather see if the current one still fits.

    For each task $v$, we first assess its current memory constraint $Res$ using Step 2 from our heuristic.
    The factors that affect $Res$ are possible changes in $m_v$, in $c_{u,v}$ from predecessors $u$ or $c_{v,w}$ from successors $w$,
    available memory $availM_j$ on the processor (due to either changed $M_j$ or changed memory requirements from other tasks).
    If originally,$Res$ was positive (no files were evicted from memory into the communication buffer), then it has to stay this way -
    otherwise evicted files can invalidate next tasks.
    If original $Res$ was negative, then we need to make sure that evicted files still fit into the communication buffer.
    If either $Res$ is newly negative, or the communication buffer is not large enough, this invalidates the schedule.
    We update the $availM_j$ and $availMC_j$ according to the new memory constraints.

    Then we can re-calculate the finish time of the task on its processor like in Step 3.
    The factors that affect it are changes in own execution time $w_v$ of the tasks, changed ready time of the processor
    (after delayed previous tasks), and changed communication buffer availability.

    Then, after having updated the processor's values, we move on to the next task.

\end{document}
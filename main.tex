\documentclass[sigconf,review,anonymous]{acmart}
\usepackage{algorithmicx}
\usepackage{amsmath}
%\usepackage[colorlinks=true, allcolors=blue]{hyperref}
\usepackage{algpseudocode}
\usepackage{algorithm}
\usepackage{xspace}

%%
%% \BibTeX command to typeset BibTeX logo in the docs
\AtBeginDocument{%
    \providecommand\BibTeX{{%
        Bib\TeX}}}

\setcopyright{acmlicensed}
\copyrightyear{2024}
\acmYear{2024}
\acmDOI{XXXXXXX.XXXXXXX}

%% These commands are for a PROCEEDINGS abstract or paper.
\acmConference[ICPP '24]{53rd International Conference on Parallel Processing}{August 12-15, 2024}{Gotland, Sweden}
\acmISBN{978-1-4503-XXXX-X/18/06}

% user-specified commands
%\newcommand{\R}{\mathbb{R}}
\newcommand{\N}{\mathbb{N}}
%\newcommand{\llminimal}{${\|\cdot\|}_2$-minimal\xspace}
\newcommand{\Pro}[1]{\mathbf{Pr} \left[\,#1\,\right]}
\newcommand{\pro}[1]{\mathbf{Pr} [\,#1\,]}
\newcommand{\Ex}[1]{\mathbb{E} \left[\,#1\,\right]}
\newcommand{\ex}[1]{\mathbb{E} [\,#1\,]}
\newcommand{\hit}{- [\pi]_s (H[v,s]-H[u,s])}

\newcommand{\Oh}{\ensuremath{\mathcal{O}}}


\newcommand{\centre}[1]{z_{#1}}
\newcommand{\centres}{Z}
\newcommand{\degree}{\operatorname{deg}}
\newcommand{\maxdeg}{\operatorname{maxdeg}}
\newcommand{\diam}{\operatorname{diam}}
\newcommand{\res}{\operatorname{res}}
\newcommand{\Res}{\operatorname{Res}}
%\newcommand{\cond}{\operatorname{cond}}
\newcommand{\Cond}{\operatorname{Cond}}
\newcommand{\proxy}{\operatorname{proxy}}
\newcommand{\excond}{\operatorname{ex\_cond}}
\newcommand{\exres}{\operatorname{ex\_res}}
\newcommand{\exalg}{\operatorname{ex\_alg}}
\newcommand{\dist}{\operatorname{dist}}
\newcommand{\cost}{\operatorname{c}}
\newcommand{\ord}{\operatorname{ord}}
\newcommand{\Vor}{\operatorname{Vor}}
\newcommand{\M}{\mathbf{M}}
\newcommand{\LL}{\mathbf{L}}
%\newcommand{\L}{\mathbf{L}}
\newcommand{\RR}{\mathbb{R}}
\newcommand{\f}{\hat{f}}
\newcommand{\e}{\mathbf{e}}
\newcommand{\dhat}{d}
\newcommand{\llminimal}{${\|\cdot\|}_2$-minimal\xspace}
\newcommand{\NP}{$\mathcal{NP}$}
\newcommand{\disjbigcup}{\mathop{\dot{\bigcup}}} 
\newcommand{\disjcup}{\mathop{\dot{\cup}}} 
\newcommand{\bigO}{\mathcal{O}} 
\newcommand{\polylog}{\operatorname{polylog}} 
\newcommand{\dotcup}{\stackrel{.}{\cup}}
\newcommand{\myvec}[1]{\mathbf{#1}}
\newcommand{\vent}[2]{\myvec{#1}[#2]}
\newcommand{\size}[1]{\operatorname{size}(#1)}
\newcommand{\distance}[2]{\operatorname{d_{G_p}}[#1][#2]}
\newcommand{\prefix}[1]{\operatorname{prefix}(#1)}
\newcommand{\clz}[1]{\operatorname{clz}(#1)}
\newcommand*\xor{\oplus}


\DeclareMathOperator{\intraWeight}{\it intraWeight}
\DeclareMathOperator{\interWeight}{\it interWeight}
\DeclareMathOperator{\subtreeVol}{\it subtreeVol}
\DeclareMathOperator{\cutWeight}{\it cutWeight}
\DeclareMathOperator{\conduct}{\it conduct}
\DeclareMathOperator{\inCutSet}{\it inCutSet}

\newcommand{\iec}{\textit{i.\,e.},\xspace}
\newcommand{\ie}{\textit{i.\,e.}\xspace}
\newcommand{\Ie}{\textit{I.\,e.}\xspace}
\newcommand{\egc}{\textit{e.\,g.},\xspace}
\newcommand{\eg}{\textit{e.\,g.}\xspace}
\newcommand{\Eg}{\textit{E.\,g.}\xspace}
\newcommand{\etal}{\textit{et al.}\xspace}
\newcommand{\Wlog}{w.\,l.\,o.\,g.\xspace}
\newcommand{\wrt}{w.\,r.\,t.\xspace}
\newcommand{\cf}{cf.\xspace}
\newcommand{\etc}{etc.\xspace}

\newcommand{\networkit}{\textsc{NetworKit}\xspace}
\newcommand{\mswap}{\textsc{TiMEr}\xspace}
\newcommand{\karma}{\textsc{KarMa}\xspace}
\newcommand{\greedyallc}{\textsc{GreedyAllC}\xspace}
\newcommand{\dibap}{\textsc{DibaP}\xspace}
\newcommand{\pdibap}{\textsc{PDibaP}\xspace}
\newcommand{\bubble}{\textsc{Bubble}\xspace}
\newcommand{\smooth}{\textsc{Smooth}}
\newcommand{\bubfosc}{\textsc{Bubble-FOS/C}\xspace}
\newcommand{\bubfost}{\textsc{Bubble-FOS/T}\xspace}
\newcommand{\metis}{\textsc{METIS}\xspace}
\newcommand{\kmetis}{\textsc{kMeTiS}\xspace}
\newcommand{\parmetis}{\textsc{ParMETIS}\xspace}
\newcommand{\libtopomap}{\textsc{LibTopoMap}\xspace}
\newcommand{\kappart}{\textsc{KaPPa}\xspace}
\newcommand{\kahip}{\textsc{KaHIP}\xspace}
\newcommand{\mapkahip}{\textsc{KaHIP\_map}\xspace}
\newcommand{\parhip}{\textsc{ParHIP}\xspace}
\newcommand{\mpipp}{\textsc{MpiPP}\xspace}
\newcommand{\jostle}{\textsc{Jostle}\xspace}
\newcommand{\zoltan}{\textsc{Zoltan}\xspace}
\newcommand{\parkway}{\textsc{Parkway}\xspace}
\newcommand{\graclus}{\textsc{Graclus}\xspace}
\newcommand{\party}{\textsc{Party}\xspace}
\newcommand{\scotch}{\textsc{Scotch}\xspace}
\newcommand{\ptscotch}{\textsc{PT-Scotch}\xspace}
\newcommand{\tree}{\textsc{Tree-Matching}\xspace}
\newcommand{\topomatch}{\textsc{TopoMatch}\xspace}
\newcommand{\xpulp}{\textsc{xtraPulp}\xspace}
\newcommand{\thrsh}{$\mathtt{thrsh}$\xspace}
\newcommand{\trunccons}{\textsc{TruncCons}\xspace}
\newcommand{\consol}{\texttt{Consolidation}\xspace}
\newcommand{\consols}{\texttt{Consolidations}\xspace}
\newcommand{\asspart}{\texttt{AssignPartition}\xspace}
\newcommand{\assclus}{\texttt{AssignCluster}\xspace}
\newcommand{\asssd}{\texttt{AssignSubdomain}\xspace}
\newcommand{\compcen}{\texttt{ComputeCenters}\xspace}
\newcommand{\initcen}{\textsc{LoadBasedInitialCenters}\xspace}
\newcommand{\NN}{\mbox{\rm I$\!$N}}
\newcommand{\closu}[1]{\overline{#1}}
\newcommand{\djoko}{Djokovi\'{c} relation\xspace}
\newcommand{\djokoRelated}{Djokovi\'{c}\xspace related\xspace}
\newcommand{\subE}[1]{{#1}_{\mathcal{E}}}
\newcommand{\comm}{\operatorname{Co}}

\newcommand{\argmin}{\operatorname{argmin}\xspace}
\newcommand{\argmax}{\operatorname{argmax}\xspace}
\newcommand{\cc}{\operatorname{Coco}\xspace}
\newcommand{\divers}{\operatorname{Div}\xspace}
\newcommand{\ccd}{\operatorname{Coco^+}\xspace}
\newcommand{\dil}{\operatorname{dil}\xspace}
\newcommand{\contract}{\operatorname{contract}\xspace}
\newcommand{\assemble}{\operatorname{assemble}\xspace}
\newcommand{\modulo}{\operatorname{mod}\xspace}

\newcommand{\parent}{\operatorname{parent}\xspace}
\newcommand{\oldParent}{\operatorname{oldParent}\xspace}
\newcommand{\newParent}{\operatorname{newParent}\xspace}
\newcommand{\newParentLabel}{\operatorname{newParentLabel}\xspace}
\newcommand{\prefLabel}{\operatorname{prefLabel}\xspace}

\newcommand{\initial}{\textsc{Identity}\xspace}
\newcommand{\initialM}{\textsc{Identity}_{\textsc{g}} \xspace}
\newcommand{\initialMO}{\textsc{Identity}_{\textsc{n}} \xspace}
\newcommand{\random}{\textsc{Random}\xspace}
\newcommand{\randomM}{\textsc{Random}_{\textsc{g}} \xspace}
\newcommand{\randomMO}{\textsc{Random}_{\textsc{n}} \xspace}
\newcommand{\rcm}{\textsc{RCM}\xspace}
\newcommand{\rcmM}{\textsc{RCM}_{\textsc{g}} \xspace}
\newcommand{\rcmMO}{\textsc{RCM}_{\textsc{n}} \xspace}
\newcommand{\durebi}{\textsc{DRB}\xspace}
\newcommand{\durebiM}{\textsc{DRB}_{\textsc{g}} \xspace}
\newcommand{\durebiMO}{\textsc{DRB}_{\textsc{n}} \xspace}
\newcommand{\greedyall}{\textsc{GreedyAll}\xspace}
\newcommand{\greedyallM}{\textsc{GreedyAll}_{\textsc{g}} \xspace}
\newcommand{\greedyallMO}{\textsc{GreedyAll}_{\textsc{n}} \xspace}
\newcommand{\greedyallcM}{\textsc{GreedyAllC} \xspace}
\newcommand{\greedyallcMO}{\textsc{greedyAllC}_{\textsc{n}} \xspace}
\newcommand{\greedymin}{\textsc{GreedyMin}\xspace}
\newcommand{\greedyminM}{\textsc{GreedyMin}_{\textsc{g}} \xspace}
\newcommand{\greedyminMO}{\textsc{GreedyMin}_{\textsc{n}} \xspace}
\newcommand{\greedyminc}{\textsc{GreedyMinC}\xspace}
\newcommand{\greedymincM}{\textsc{GreedyMinC}_{\textsc{g}} \xspace}
\newcommand{\greedymincMO}{\textsc{GreedyMinC}_{\textsc{n}} \xspace}
\newcommand{\walshawlarge}{\textsc{WalshawLarge}\xspace}
\newcommand{\complexnets}{\textsc{ComplexNets}\xspace}
\newcommand{\gpmetis}{\textsc{gpMetis}\xspace}
\newcommand{\ndmetis}{\textsc{ndMetis}\xspace}

\newcommand{\seacode}{\textsc{SEAmap}\xspace}
\newcommand{\ouralgo}{\textsc{ParHIP\_map}\xspace }


\newcommand{\real}{\textsc{real}\xspace}
\newcommand{\ba}{\textsc{BA}\xspace}
\newcommand{\rhg}{\textsc{RHG}\xspace}
\newcommand{\rmat}{\textsc{Rmat}\xspace}


%% \newcommand{\cone}{\texttt{cSc}\xspace}
%% \newcommand{\ctwo}{\texttt{cId}\xspace}
%% \newcommand{\cthree}{\texttt{cGr}\xspace}
%% \newcommand{\cfour}{\texttt{cLb}\xspace}
\newcommand{\cone}{\mathtt{c1}\xspace}
\newcommand{\ctwo}{\mathtt{c2}\xspace}
\newcommand{\cthree}{\mathtt{c3}\xspace}
\newcommand{\cfour}{\mathtt{c4}\xspace}



\newcommand{\algo}[1]{\textsc{#1}}
\newcommand{\bottomlevel}[1]{\underline{l}_{#1}} % underline short italic
\newcommand{\criticalpath}{\mathcal{P}}
\newcommand{\parents}[1]{\,\Pi_{#1}}
\newcommand{\children}[1]{\,C_{#1}}
\newcommand{\cluster}{\,\mathcal{S}}
\newcommand{\daghetpart}{\algo{DagHetPart}\xspace}
\newcommand{\dagmem}{\algo{DagHetMem}\xspace}
%%
%% end of the preamble, start of the body of the document source.
\begin{document}


%%
%% The "title" command has an optional parameter,
%% allowing the author to define a "short title" to be used in page headers.
    \title{Adaptive Scheduling of Scientific Workflows}

%%
%% The "author" command and its associated commands are used to define
%% the authors and their affiliations.
%% Of note is the shared affiliation of the first two authors, and the
%% "authornote" and "authornotemark" commands
%% used to denote shared contribution to the research.
    \author{Svetlana Kulagina}
    \email{svetlana.kulagina@hu-berlin.de}
    \orcid{0000-0002-2108-9425}
    \affiliation{%
        \institution{Humboldt Universitaet zu Berlin}
        \streetaddress{Unter den Linden 6}
        \city{Berlin}
        \country{Germany}
        \postcode{10099}
    }

       \author{Henning Meyerhenke}
       \affiliation{%
           \institution{Humboldt Universitaet zu Berlin}
           \city{Berlin}
           \country{Germany}
       }
       \email{meyerhenke@hu-berlin.de}

    \author{Anne Benoit}
    \affiliation{%
        \institution{ENS Lyon}
        \city{Lyon}
        \country{France}
    }
    \email{Anne.Benoit@ens-lyon.fr}

%%
%% By default, the full list of authors will be used in the page
%% headers. Often, this list is too long, and will overlap
%% other information printed in the page headers. This command allows
%% the author to define a more concise list
%% of authors' names for this purpose.
    \renewcommand{\shortauthors}{Trovato et al.}

%%
%% The abstract is a short summary of the work to be presented in the
%% article.
    \begin{abstract}
       Todo: reinsert CCSXML concepts
    \end{abstract}

%%
%% The code below is generated by the tool at http://dl.acm.org/ccs.cfm.
%% Please copy and paste the code instead of the example below.
%%


%%
%% Keywords. The author(s) should pick words that accurately describe
%% the work being presented. Separate the keywords with commas.
    \keywords{Scheduling, Adaptive Scheduling, DAG}
    \received{15 April 2024}
    \received[revised]{12 March 2009}
    \received[accepted]{5 June 2009}

%%
%% This command processes the author and affiliation and title
%% information and builds the first part of the formatted document.
    \maketitle

    \section{Introduction}


    \section{Model}

    \begin{table}[h]
        \begin{center}
            \begin{tabular}{rl}
                \hline
                \textbf{Symbol} & \textbf{Meaning}  \\
                \hline
                $G = (V, E)$  & Workflow graph, set of vertices (tasks) and edges  \\
                $\parents{u}$, $\children{u}$ & Parents of a task $u$, children of a task $u$ \\
                $m_u$& Memory weight of task $u$ \\
                $w_u$   & Workload of  task $u$  (normalized execution time)    \\
                $c_{u,v}$   & Communication volume along the edge $(u,v)\in E$ \\
                $F$, $\mathcal{F}$ & A partitioning function and the partition it creates \\
                $V_i$ & Block number $i$\\ %\wrt~some $F$   \\
                $\cluster$, $k$   & Computing system and its number of processors   \\
                $p_j$, proc($V_i$)  & Processor number $j$, processor of block $V_i$ \\
                $M_j$, $s_j$   & Memory size and speed of processor $p_j$   \\
                $\beta$ & Bandwidth in the compute system  \\
                $\bottomlevel{u}$   & Bottom weight of task $u$   \\
                $\mu_G$, $\mu_i$ & Makespan of the entire workflow $G$ and of a block $V_i$ \\
                $\Gamma = (\mathcal{V}, \mathcal{E})$ & Quotient graph, its vertices and its edges  \\
                $r_u$, $r_{V_i}$  & Memory requirement of  task $u$ and of block $V_i$   \\
                $r_{\max}$   & Maximum memory requirement in a workflow   \\
                $\criticalpath$ & Critical path in a workflow  \\
                \hline
            \end{tabular}
        \end{center}
        \caption{Notation.} \label{tabnotation}
    \end{table}

    \paragraph{Workflow-related changes}

    \begin{itemize}
        \item A task $v$ takes longer or shorter to execute than planned: its time weight $w_u$ changes to $w'_u$.
        \item A task $v$ takes more or less memory to execute than planned: its memory requirement $m_v$ changes to $m'_v$.

    \end{itemize}

    The following changes are not a part of this article's scope:

    \begin{itemize}
        \item The workflow structure changes: edges or tasks come in or leave.
    \end{itemize}

    \paragraph{Execution environment-related changes }


    \begin{itemize}
        \item A processor exists the execution environment: $k$ decreases and $\cluster$ changes.
        \item A processor enters the execution environment: $k$ increases, $\cluster$ gets a new processor with possibly new memory requirement and processor speed.

    \end{itemize}

    The following changes are not a part of this article's scope:

    \begin{itemize}
        \item Processor characteristics change: the memory requirement or speed become bigger or smaller
    \end{itemize}

    \subsection{Time of changes }

    We consider discrete time in seconds.
    The time point(s) at which the changes happen is unambiguously defined.

    For any task $v$, its runtime equals its time weight divided by the speed of the processor $p_j$ it has been assigned to: $w_v/s_j$.
    The start time of any task $v$ is its top level($\bar{l}_v$), or the difference between the maximum bottom level in the workflow (the makespan of the workflow) and the task's own bottom level: $\bar{l}_v = \mu_\Gamma - \bottomlevel{v}$.
    The start time of the source task in the workflow is zero.
    The end time of a task $v$ is its start time and its runtime: $\bar{l}_v + w_v/s_j$

    \subsection{Changes and knowledge horizon - important questions TBA}

    Given a valid mapping of tasks to processors, we can say what we predicted would happen at any given time point $T$: what tasks have been executed, what have not finished or have not even started.

    At the point of change, we know that some tasks that finished took longer than expected ($w_v$ are bigger) or shorter.
    However, how do we model the following:
    \begin{itemize}
        \item Do we know the new weights of currently running tasks and tasks that have not yet started? This means, do we foresee into the future or do we assume that all weights on unfinished tasks remain the same?
        \item A change in memory requirements can mean that the assignment had been invalid. Do we assume that these tasks failed and we need to rerun them?
        \item How many times of change do we model - one per workflow run, or multiple?
        \item At what time does the change and reevaluation happen - is it a fixed (random?) point of time or is it workflow-dependent (say, after 10\% of the workflow is ready)?
    \end{itemize}

    \section{Related work}

    In the limited time I spent looking, I found no paper addressing the exact same problem.

    Wang et al.~\cite{wang2019dynamic} proposes a dynamic particle swarm optimization algorithm to schedule workflows in a cloud.
    Particles are possible solution in the solution space.
    However, the dynamic is only in the choice of generation sizes, not in the changes in the execution environment.
    Singh et al.~\cite{singh2018novel} addresses dynamic provisioning of resources with a constraint deadline.
    However, the approach is for clouds.
    \xspace

    Daniels et al.

    De Olivera~\etal~\cite{de2012provenance} propose a tri-criteria (makespan, reliability, cost) adaptive scheduling heuristic
    for clouds.
    Based on a 3-objective cost model, their greedy scheduling algorithm schedules tasks into machines.
    The authors use provenance data to make scheduling decisions.
    The cost model is a set of linear equations computed in the simulation environment, it represents the cost of an execution based on the criteria.
    In the algorithms, the authors test out 4 scenarios - one preferring each criteria, and a balanced one.
    The algorithm merely chooses the best virtual machine for each next task based on the cost given by the model.
    An additional algorithm combines several cloud activities (task executions) into one to improve the cost, so that each
    execution is not too small (utilize the granularity factor, the smallest entity of payment of the cloud producer).
    The authors used workflows with less than 10 tasks, but repeated them so that the execution had up to 200 tasks.
    They do not report the runtime of the scheduling algorithm, only the speedup and cost saving it produces.

    Rahman~\etal~\cite{rahman2013} propose a scheduling heuristic for grids that proposes mapping of tasks to machines by calculating
    the critical path in the graph dynamically at every step.
    They call it the dynamic critical path (DCP).
    For all tasks they compute the earliest start time and absolute latest start time that are upper and lower bounds
    on the start time of a task (differing by the slack this task has).
    All tasks on this critical path have the same earliest and latest start times, because they cannot be ddelayed.

    The algorithm takes the first unscheduled task on the critical path each time and maps it on a processor identified for it.
    If processors are heterogeneous, then the start times are computed with respect for the processor, and the minimum execution time for
    the task is chosen.
    The heuristic also uses the same processor to schedule parent and children tasks, as to avoid data transfer between processors.
    The authors evaluate their approaches on random workflows of the size up to 300 tasks.

    The authors provide an overview over (simpler) scheduling heuristics.
    For example, GRASP (generally randomized adaptive search procedure) conducts a number of iterations to search an optimal
    solution for mapping tasks on machines.
    A solution is generated at each step, and the best solution is kept at the end.
    The search terminates when a certain termination criterion is reached.
    It generates better results than other algorithms, because it explores the whole solution space.

    Avanes~\etal\cite{avanes2008adaptive} present a heuristic for networks in disaster scenarios.
    These networks are a set of DAG-shaped scenarios, out of which one needs to be executed.
    The scenario contains AND- and OR-branches, where AND-branches indicate acitivities that need to be executed in parallel.

    The heuristic first partitions the set into local schedules by using affinity matrices to determine similar
    activities and group them together.
    Then they physically allocate these partitions to groups of disaster responders and tasks within this group.
    They define a constraint system for that.
    The dynamic part deals with changes and distinguishes between retriable and compensation acitivities.
    The heuristic calculates a new execution path with these tasks.
    In general, this heuristic is not as related as it looks, because of very specific workflow and task structure.




    \section{Algorithms}

    \subsection{Various useful approaches}

    Bader et al. developed a system that identifies task similarities.
    Given a workflow, it returns tasks similar to a given task.
    We can use this system in our modelling and say that we know the new weights for finished tasks and will modify future tasks in accordance to their similarity.
    For example, if a task finished 2 times later than expected, then all future similar tasks will be assumed to finish 2 times later.

    \subsection{Local search}
    One strategy that can be used is local search, a similar approach as with swaps.
    The starting point could be a previously valid solution (for example, one by HetPart).
    In the first step, the solution would need to become valid again.
    Then, we repartition blocks on the critical path  and merge blocks in attempt to improve makespan.
    Finally, we swap blocks between processors to finish makespan improvement.

    \subsubsection{Restore Validity}
    We~(Algorithm~\ref{alg:RestoreValidity}) start with the previously partitioned and assigned quotient tree $\Gamma$,
    and the new weights on tasks and edges.
    We first recompute the memory requirements of all blocks.
    Then we identify all blocks that have become invalidated and exceed the memory of their assigned processor
    (Line~\ref{line:damaged}).
    For all these blocks, we try out ``border'' tasks, that is, tasks, that have neighbours in neighbouring blocks
    (Line~\ref{line:neighb}).
    We tentatively move them (creating a temporary quotient DAG $\Gamma'$) and assess the impact of this move on the
    makespan(Line~\ref{line:makespan}).
    The condition is, of course, that the neighboring block now does not exceed the memory of its own processor.
    The task that delivers the best makespan in then moved in Line~\ref{line:movebest}.
    The entire process is being repeated until the block fits into the memory of the processor again.
    We return the modified quotient graph $\Gamma$ that contains the mapping to processors.

    \begin{algorithm}[h!]
        \caption{Restore Validity}
        \label{alg:RestoreValidity}
        \begin{algorithmic}[1]
            \Procedure{Restore Validity}{$\Gamma$, $W$ }\\
            \Comment{Input: initial schedule as quotient tree $\Gamma$, new  weights $W$}
            \State $\Gamma \gets $ \textsc{QuotientDAG}($\Gamma$, $W$ ); \Comment Quotient DAG with new weights

            \State $D \gets \emptyset$ \Comment {Damaged blocks that require action}
            \For{$\nu_i \in \Gamma$}
                \State $r_{\nu_m} \gets W$
                \Comment{Recompute the new memory requirements of blocks in the quotient DAG}
            \EndFor
            \For{$\nu_i \in \Gamma$}
                \label{line:damaged}
                \If{$r_{\nu_i} > M_{proc(\nu_i)}$}
                    \State $D \gets D \cup \{\nu_i\}$
                \EndIf
            \EndFor

            \For{$\nu_i \in D$}
                \Comment Restore the validity of each damaged block
                \While{$r_{\nu_i} > M_{proc(\nu_i)}$}
                    \State $MS_{min}\gets \infty; v_{min} \gets NULL; \nu_{min} \gets NULL$
                    \Comment Moving $v_{min}$ to $\nu_{min}$ gives $MS_{min}$
                    \For{$v \in \nu_i$}
                        \For{$u \in ( \Pi_v \cup C_v) \cap (\Pi_{\nu_i} \cup C_{\nu_i})$}   \label{line:neighb}
                        \State $\Gamma' \gets \Gamma \mid \nu(v) == \nu(u)$
                        \If{ $r_{\nu(u)} > M_{proc(\nu(u))}$ and \textsc{Makespan}($\Gamma'$)$\leq MS_{min}$}\label{line:makespan}
                        \State $MS_{min} \gets $\textsc{Makespan}($\Gamma'$); $v_{min} \gets v$; $\nu_{min} \gets \nu(u);$
                        \EndIf
                        \EndFor
                    \EndFor
                    \State  $\Gamma \gets \Gamma \mid \nu(v_{min}) == \nu_{min}$ \label{line:movebest}
                \EndWhile
            \EndFor
            \State return $\Gamma$
            \EndProcedure
        \end{algorithmic}
    \end{algorithm}


    \subsubsection{Repartition and Merge}

    In the next step, we center our attention around changes in the execution environments and how the changes in
    the workflow reflect on it.

    First, we try to further partition blocks that lie on the critical path of the workflow.
    We first compute the critical path (Line~\ref{line:cp}) and then, as long as there are free processors in the
    execution environment(Line~\ref{line:whilefree}, assumed to be sorted in descending order of memory sizes),
    we partition the blocks on the critical path(Line~\ref{line:repartition}).
    We try to partition each block into 2 smaller blocks.
    However, due to the nature of the partitioner, more can be created.
    We assume that the resulting blocks(Line~\ref{line:repartition}) are given back sorted by their memory requirement.
    We then assign them to free processors as long as the blocks fit and there are free processors.
    If even one part does not fit(Line~\ref{line:ifnotfits}), then we discard the entire partition and try partitioning
    the next block on the critical path.

    Not only partitioning (introducing more parallelism) can lead to a makespan improvement.
    Shorter runtimes of tasks can justify merging some blocks together.
    For all pairs of blocks that are neighbours (Line~\ref{Line:neighbblocks}), we tentatively merge them and assign
    the result to the processor of the first block.
    If the merged block fits on this processor and the merged quotient graph has a smaller overall makespan, then we save
    the merge and proceed with other blocks.

    \begin{algorithm}[h!]
        \caption{Repartition and Merge}
        \label{alg:RepartitionMerge}
        \begin{algorithmic}[1]
            \Procedure{Repartition and Merge}{$\Gamma$, $\cluster$ }\\
            \Comment{Input: quotient tree $\Gamma$, execution environment $\cluster$}
            \State $CP \gets $\textsc{Criticalpath}($\Gamma$) \label{line:cp}
            \State $Free \gets \{ p \in \cluster \mid p$ is free$\}$
            \While{$\mid Free \mid \neq 0$}\label{line:whilefree}
            \For{$\nu \in CP$}
                \State $\{ \nu_1, \nu_2,\dots\}\gets$\textsc{Partition}($\nu$,2) \label{line:repartition}
                \For{$\nu' \in \{ \nu_1, \nu_2,\dots\}$}
                    \If{$r_\nu' \leq Free.$head()}
                        \State proc($\nu'$)$\gets Free.$head();
                    \Else \label{line:ifnotfits}
                    \State \textsc{Merge}$\nu' \in \{ \nu_1, \nu_2,\dots\}$; break;
                    \Comment Merge the parts back and leave this partitioning, as it does not fit
                    \EndIf
                \EndFor
            \EndFor
            \EndWhile

            \For{$\nu_1, \nu_2 \in \Gamma \mid \nu_1 \in ( \Pi_{\nu_2} \cup C_{\nu_2})$}
                \label{line:neighbblocks}
            \State $\{\Gamma', \nu_m\} \gets $ \textsc{Merge}($\nu_1, \nu_2$); proc($\nu_m$)$\gets $ proc($\nu_1$);
            \State $\mu \gets $\textsc{Makespan}($\Gamma'$)
            \If{$r_{\nu_m} \leq M_{proc(\nu_m)}$ and $\mu \leq $ \textsc{Makespan}($\Gamma$)}
                \State $\Gamma \gets \Gamma'$
            \EndIf
            \EndFor

            \State return $\Gamma$
            \EndProcedure
        \end{algorithmic}
    \end{algorithm}



    \subsubsection{Swaps}

    Then a process similar to Step 4 in %\daghetpart
    can be executed: the newly rearranged blocks are swapped until no
    improvement in makespan can be achieved.

    \subsection{Limitations}

    We assume that moving a task from one processor to another costs nothing, when in reality it requires a copy
    operation that can take time at least equal to the sum of the weights of all its input files divided by the bandwidth.

    \bibliographystyle{ACM-Reference-Format}
    \bibliography{references}

\end{document}
\endinput
%%
%% End of file `sample-sigconf.tex'.
\documentclass[conference]{IEEEtran}
\IEEEoverridecommandlockouts
% The preceding line is only needed to identify funding in the first footnote. If that is unneeded, please comment it out.
\usepackage{cite}
\usepackage{amsmath,amssymb,amsfonts}
%\usepackage{algorithmic}
\usepackage{graphicx}
\usepackage{textcomp}
\usepackage{xcolor}
\usepackage{algorithmicx}
%\usepackage[colorlinks=true, allcolors=blue]{hyperref}
\usepackage{algpseudocode}
\usepackage{algorithm}
\usepackage{xspace}

\def\BibTeX{{\rm B\kern-.05em{\sc i\kern-.025em b}\kern-.08em
T\kern-.1667em\lower.7ex\hbox{E}\kern-.125emX}}
% user-specified commands
%\newcommand{\R}{\mathbb{R}}
\newcommand{\N}{\mathbb{N}}
%\newcommand{\llminimal}{${\|\cdot\|}_2$-minimal\xspace}
\newcommand{\Pro}[1]{\mathbf{Pr} \left[\,#1\,\right]}
\newcommand{\pro}[1]{\mathbf{Pr} [\,#1\,]}
\newcommand{\Ex}[1]{\mathbb{E} \left[\,#1\,\right]}
\newcommand{\ex}[1]{\mathbb{E} [\,#1\,]}
\newcommand{\hit}{- [\pi]_s (H[v,s]-H[u,s])}

\newcommand{\Oh}{\ensuremath{\mathcal{O}}}


\newcommand{\centre}[1]{z_{#1}}
\newcommand{\centres}{Z}
\newcommand{\degree}{\operatorname{deg}}
\newcommand{\maxdeg}{\operatorname{maxdeg}}
\newcommand{\diam}{\operatorname{diam}}
\newcommand{\res}{\operatorname{res}}
\newcommand{\Res}{\operatorname{Res}}
%\newcommand{\cond}{\operatorname{cond}}
\newcommand{\Cond}{\operatorname{Cond}}
\newcommand{\proxy}{\operatorname{proxy}}
\newcommand{\excond}{\operatorname{ex\_cond}}
\newcommand{\exres}{\operatorname{ex\_res}}
\newcommand{\exalg}{\operatorname{ex\_alg}}
\newcommand{\dist}{\operatorname{dist}}
\newcommand{\cost}{\operatorname{c}}
\newcommand{\ord}{\operatorname{ord}}
\newcommand{\Vor}{\operatorname{Vor}}
\newcommand{\M}{\mathbf{M}}
\newcommand{\LL}{\mathbf{L}}
%\newcommand{\L}{\mathbf{L}}
\newcommand{\RR}{\mathbb{R}}
\newcommand{\f}{\hat{f}}
\newcommand{\e}{\mathbf{e}}
\newcommand{\dhat}{d}
\newcommand{\llminimal}{${\|\cdot\|}_2$-minimal\xspace}
\newcommand{\NP}{$\mathcal{NP}$}
\newcommand{\disjbigcup}{\mathop{\dot{\bigcup}}} 
\newcommand{\disjcup}{\mathop{\dot{\cup}}} 
\newcommand{\bigO}{\mathcal{O}} 
\newcommand{\polylog}{\operatorname{polylog}} 
\newcommand{\dotcup}{\stackrel{.}{\cup}}
\newcommand{\myvec}[1]{\mathbf{#1}}
\newcommand{\vent}[2]{\myvec{#1}[#2]}
\newcommand{\size}[1]{\operatorname{size}(#1)}
\newcommand{\distance}[2]{\operatorname{d_{G_p}}[#1][#2]}
\newcommand{\prefix}[1]{\operatorname{prefix}(#1)}
\newcommand{\clz}[1]{\operatorname{clz}(#1)}
\newcommand*\xor{\oplus}


\DeclareMathOperator{\intraWeight}{\it intraWeight}
\DeclareMathOperator{\interWeight}{\it interWeight}
\DeclareMathOperator{\subtreeVol}{\it subtreeVol}
\DeclareMathOperator{\cutWeight}{\it cutWeight}
\DeclareMathOperator{\conduct}{\it conduct}
\DeclareMathOperator{\inCutSet}{\it inCutSet}

\newcommand{\iec}{\textit{i.\,e.},\xspace}
\newcommand{\ie}{\textit{i.\,e.}\xspace}
\newcommand{\Ie}{\textit{I.\,e.}\xspace}
\newcommand{\egc}{\textit{e.\,g.},\xspace}
\newcommand{\eg}{\textit{e.\,g.}\xspace}
\newcommand{\Eg}{\textit{E.\,g.}\xspace}
\newcommand{\etal}{\textit{et al.}\xspace}
\newcommand{\Wlog}{w.\,l.\,o.\,g.\xspace}
\newcommand{\wrt}{w.\,r.\,t.\xspace}
\newcommand{\cf}{cf.\xspace}
\newcommand{\etc}{etc.\xspace}

\newcommand{\networkit}{\textsc{NetworKit}\xspace}
\newcommand{\mswap}{\textsc{TiMEr}\xspace}
\newcommand{\karma}{\textsc{KarMa}\xspace}
\newcommand{\greedyallc}{\textsc{GreedyAllC}\xspace}
\newcommand{\dibap}{\textsc{DibaP}\xspace}
\newcommand{\pdibap}{\textsc{PDibaP}\xspace}
\newcommand{\bubble}{\textsc{Bubble}\xspace}
\newcommand{\smooth}{\textsc{Smooth}}
\newcommand{\bubfosc}{\textsc{Bubble-FOS/C}\xspace}
\newcommand{\bubfost}{\textsc{Bubble-FOS/T}\xspace}
\newcommand{\metis}{\textsc{METIS}\xspace}
\newcommand{\kmetis}{\textsc{kMeTiS}\xspace}
\newcommand{\parmetis}{\textsc{ParMETIS}\xspace}
\newcommand{\libtopomap}{\textsc{LibTopoMap}\xspace}
\newcommand{\kappart}{\textsc{KaPPa}\xspace}
\newcommand{\kahip}{\textsc{KaHIP}\xspace}
\newcommand{\mapkahip}{\textsc{KaHIP\_map}\xspace}
\newcommand{\parhip}{\textsc{ParHIP}\xspace}
\newcommand{\mpipp}{\textsc{MpiPP}\xspace}
\newcommand{\jostle}{\textsc{Jostle}\xspace}
\newcommand{\zoltan}{\textsc{Zoltan}\xspace}
\newcommand{\parkway}{\textsc{Parkway}\xspace}
\newcommand{\graclus}{\textsc{Graclus}\xspace}
\newcommand{\party}{\textsc{Party}\xspace}
\newcommand{\scotch}{\textsc{Scotch}\xspace}
\newcommand{\ptscotch}{\textsc{PT-Scotch}\xspace}
\newcommand{\tree}{\textsc{Tree-Matching}\xspace}
\newcommand{\topomatch}{\textsc{TopoMatch}\xspace}
\newcommand{\xpulp}{\textsc{xtraPulp}\xspace}
\newcommand{\thrsh}{$\mathtt{thrsh}$\xspace}
\newcommand{\trunccons}{\textsc{TruncCons}\xspace}
\newcommand{\consol}{\texttt{Consolidation}\xspace}
\newcommand{\consols}{\texttt{Consolidations}\xspace}
\newcommand{\asspart}{\texttt{AssignPartition}\xspace}
\newcommand{\assclus}{\texttt{AssignCluster}\xspace}
\newcommand{\asssd}{\texttt{AssignSubdomain}\xspace}
\newcommand{\compcen}{\texttt{ComputeCenters}\xspace}
\newcommand{\initcen}{\textsc{LoadBasedInitialCenters}\xspace}
\newcommand{\NN}{\mbox{\rm I$\!$N}}
\newcommand{\closu}[1]{\overline{#1}}
\newcommand{\djoko}{Djokovi\'{c} relation\xspace}
\newcommand{\djokoRelated}{Djokovi\'{c}\xspace related\xspace}
\newcommand{\subE}[1]{{#1}_{\mathcal{E}}}
\newcommand{\comm}{\operatorname{Co}}

\newcommand{\argmin}{\operatorname{argmin}\xspace}
\newcommand{\argmax}{\operatorname{argmax}\xspace}
\newcommand{\cc}{\operatorname{Coco}\xspace}
\newcommand{\divers}{\operatorname{Div}\xspace}
\newcommand{\ccd}{\operatorname{Coco^+}\xspace}
\newcommand{\dil}{\operatorname{dil}\xspace}
\newcommand{\contract}{\operatorname{contract}\xspace}
\newcommand{\assemble}{\operatorname{assemble}\xspace}
\newcommand{\modulo}{\operatorname{mod}\xspace}

\newcommand{\parent}{\operatorname{parent}\xspace}
\newcommand{\oldParent}{\operatorname{oldParent}\xspace}
\newcommand{\newParent}{\operatorname{newParent}\xspace}
\newcommand{\newParentLabel}{\operatorname{newParentLabel}\xspace}
\newcommand{\prefLabel}{\operatorname{prefLabel}\xspace}

\newcommand{\initial}{\textsc{Identity}\xspace}
\newcommand{\initialM}{\textsc{Identity}_{\textsc{g}} \xspace}
\newcommand{\initialMO}{\textsc{Identity}_{\textsc{n}} \xspace}
\newcommand{\random}{\textsc{Random}\xspace}
\newcommand{\randomM}{\textsc{Random}_{\textsc{g}} \xspace}
\newcommand{\randomMO}{\textsc{Random}_{\textsc{n}} \xspace}
\newcommand{\rcm}{\textsc{RCM}\xspace}
\newcommand{\rcmM}{\textsc{RCM}_{\textsc{g}} \xspace}
\newcommand{\rcmMO}{\textsc{RCM}_{\textsc{n}} \xspace}
\newcommand{\durebi}{\textsc{DRB}\xspace}
\newcommand{\durebiM}{\textsc{DRB}_{\textsc{g}} \xspace}
\newcommand{\durebiMO}{\textsc{DRB}_{\textsc{n}} \xspace}
\newcommand{\greedyall}{\textsc{GreedyAll}\xspace}
\newcommand{\greedyallM}{\textsc{GreedyAll}_{\textsc{g}} \xspace}
\newcommand{\greedyallMO}{\textsc{GreedyAll}_{\textsc{n}} \xspace}
\newcommand{\greedyallcM}{\textsc{GreedyAllC} \xspace}
\newcommand{\greedyallcMO}{\textsc{greedyAllC}_{\textsc{n}} \xspace}
\newcommand{\greedymin}{\textsc{GreedyMin}\xspace}
\newcommand{\greedyminM}{\textsc{GreedyMin}_{\textsc{g}} \xspace}
\newcommand{\greedyminMO}{\textsc{GreedyMin}_{\textsc{n}} \xspace}
\newcommand{\greedyminc}{\textsc{GreedyMinC}\xspace}
\newcommand{\greedymincM}{\textsc{GreedyMinC}_{\textsc{g}} \xspace}
\newcommand{\greedymincMO}{\textsc{GreedyMinC}_{\textsc{n}} \xspace}
\newcommand{\walshawlarge}{\textsc{WalshawLarge}\xspace}
\newcommand{\complexnets}{\textsc{ComplexNets}\xspace}
\newcommand{\gpmetis}{\textsc{gpMetis}\xspace}
\newcommand{\ndmetis}{\textsc{ndMetis}\xspace}

\newcommand{\seacode}{\textsc{SEAmap}\xspace}
\newcommand{\ouralgo}{\textsc{ParHIP\_map}\xspace }


\newcommand{\real}{\textsc{real}\xspace}
\newcommand{\ba}{\textsc{BA}\xspace}
\newcommand{\rhg}{\textsc{RHG}\xspace}
\newcommand{\rmat}{\textsc{Rmat}\xspace}


%% \newcommand{\cone}{\texttt{cSc}\xspace}
%% \newcommand{\ctwo}{\texttt{cId}\xspace}
%% \newcommand{\cthree}{\texttt{cGr}\xspace}
%% \newcommand{\cfour}{\texttt{cLb}\xspace}
\newcommand{\cone}{\mathtt{c1}\xspace}
\newcommand{\ctwo}{\mathtt{c2}\xspace}
\newcommand{\cthree}{\mathtt{c3}\xspace}
\newcommand{\cfour}{\mathtt{c4}\xspace}



\newcommand{\algo}[1]{\textsc{#1}}
\newcommand{\bottomlevel}[1]{\underline{l}_{#1}} % underline short italic
\newcommand{\criticalpath}{\mathcal{P}}
\newcommand{\parents}[1]{\,\Pi_{#1}}
\newcommand{\children}[1]{\,C_{#1}}
\newcommand{\cluster}{\,\mathcal{S}}
\newcommand{\daghetpart}{\algo{DagHetPart}\xspace}
\newcommand{\dagmem}{\algo{DagHetMem}\xspace}

\newcommand{\MM}{M}
\newcommand{\MC}{MC}
\newcommand{\rt}{rt}
\newcommand{\curM}{curM}
\newcommand{\curC}{curC}
\newcommand{\PD}{PD}

\newcommand{\skug}[1]{{\color{blue}[SK: #1]}}
\newcommand{\hmey}[1]{{\color{red}[HM: #1]}}
\newcommand{\AB}[1]{{\color{purple}[AB: #1]}}

\begin{document}

    \title{Adaptive Scheduling of Scientific Workflows\\
%{\footnotesize \textsuperscript{*}Note: Sub-titles are not captured in Xplore and
%should not be used}
    \thanks{Identify applicable funding agency here. If none, delete this.}
    }

%\author{\IEEEauthorblockN{1\textsuperscript{st} Given Name Surname}
%\IEEEauthorblockA{\textit{dept. name of organization (of Aff.)} \\
%\textit{name of organization (of Aff.)}\\
%City, Country \\
%email address or ORCID}
%\and
%\IEEEauthorblockN{2\textsuperscript{nd} Given Name Surname}
%\IEEEauthorblockA{\textit{dept. name of organization (of Aff.)} \\
%\textit{name of organization (of Aff.)}\\
%City, Country \\
%email address or ORCID} }


    \maketitle

    \begin{abstract}
        TODO: Insert abstract
    \end{abstract}

    \begin{IEEEkeywords}
        Scheduling, Adaptive Scheduling, DAG
    \end{IEEEkeywords}

    \section{Introduction}


    \section{Model}

    \begin{table}[h]
        \begin{center}
            \begin{tabular}{rl}
                \hline
                \textbf{Symbol}                       & \textbf{Meaning}                                         \\
                \hline
                $G = (V, E)$                          & Workflow graph, set of vertices (tasks) and edges        \\
                $\parents{u}$, $\children{u}$         & Parents of a task $u$, children of a task $u$            \\
                $m_u$                                 & Memory weight of task $u$                                \\
                $w_u$                                 & Workload of task $u$  (normalized execution time)          \\
                $c_{u,v}$                             & Communication volume along the edge $(u,v)\in E$         \\
                $F$, $\mathcal{F}$                    & A partitioning function and the partition it creates     \\
                $V_i$                                 & Block number $i$                                         \\ %\wrt~some $F$   \\
                $\cluster$, $k$                    & Computing system and its number of processors           \\
                $p_j$, proc($V_i$)                          & Processor number $j$, processor of block $V_i$                 \\
                $M_j$, $s_j$                               & Memory size and speed of processor $p_j$                          \\
                $\beta$                     & Bandwidth in the compute system                                \\
                $\bottomlevel{u}$                      & Bottom weight of task $u$ \\
                $\mu_G$, $\mu_i$ & Makespan of the entire workflow $G$ and of a block $V_i$               \\
                $\Gamma = (\mathcal{V}, \mathcal{E})$                      & Quotient graph, its vertices and its edges        \\
                $r_u$, $r_{V_i}$                            & Memory requirement of task $u$ and of block $V_i$                 \\
                $r_{\max}$                       & Maximum memory requirement in a workflow                              \\
                $\criticalpath$                       & Critical path in a workflow                              \\
                \hline
            \end{tabular}
        \end{center}
        \caption{Notation.} \label{tabnotation}
    \end{table}


    \section{Related work}

    In the limited time I spent looking, I found no paper addressing the exact same problem.

    Wang et al.~\cite{wang2019dynamic} proposes a dynamic particle swarm optimization algorithm to schedule workflows in a cloud.
    Particles are possible solution in the solution space.
    However, the dynamic is only in the choice of generation sizes, not in the changes in the execution environment.
    Singh et al.~\cite{singh2018novel} addresses dynamic provisioning of resources with a constraint deadline.
    However, the approach is for clouds.
    \xspace

    Daniels et al.~\cite{daniels1995robust} formalize the concept of robust scheduling with variable processing times
    on a single machine.
    The changes in runtimes of tasks are not due to changing machine properties, but are rather task-related (that means
    that these runtime changes are unrelated to each other).
    The authors formulate a decision space of all permutations of n jobs, and the optimal schedule in relation to a
    performance measure $\phi$.
    Then they proceed to formulate the Absolute Deviation Robust Scheduling Problem as a set of linear constraints.

    De Olivera~\etal~\cite{de2012provenance} propose a tri-criteria (makespan, reliability, cost) adaptive scheduling heuristic
    for clouds.
    Based on a 3-objective cost model, their greedy scheduling algorithm schedules tasks into machines.
    The authors use provenance data to make scheduling decisions.
    The cost model is a set of linear equations computed in the simulation environment, it represents the cost of an execution based on the criteria.
    In the algorithms, the authors test out 4 scenarios - one preferring each criteria, and a balanced one.
    The algorithm merely chooses the best virtual machine for each next task based on the cost given by the model.
    An additional algorithm combines several cloud activities (task executions) into one to improve the cost, so that each
    execution is not too small (utilize the granularity factor, the smallest entity of payment of the cloud producer).
    The authors used workflows with less than 10 tasks, but repeated them so that the execution had up to 200 tasks.
    They do not report the runtime of the scheduling algorithm, only the speedup and cost saving it produces.

    Rahman~\etal~\cite{rahman2013} propose a scheduling heuristic for grids that proposes mapping of tasks to machines by calculating
    the critical path in the graph dynamically at every step.
    They call it the dynamic critical path (DCP).
    For all tasks they compute the earliest start time and absolute latest start time that are upper and lower bounds
    on the start time of a task (differing by the slack this task has).
    All tasks on this critical path have the same earliest and latest start times, because they cannot be ddelayed.

    The algorithm takes the first unscheduled task on the critical path each time and maps it on a processor identified for it.
    If processors are heterogeneous, then the start times are computed with respect for the processor, and the minimum execution time for
    the task is chosen.
    The heuristic also uses the same processor to schedule parent and children tasks, as to avoid data transfer between processors.
    The authors evaluate their approaches on random workflows of the size up to 300 tasks.

    The authors provide an overview over (simpler) scheduling heuristics.
    For example, GRASP (generally randomized adaptive search procedure) conducts a number of iterations to search an optimal
    solution for mapping tasks on machines.
    A solution is generated at each step, and the best solution is kept at the end.
    The search terminates when a certain termination criterion is reached.
    It generates better results than other algorithms, because it explores the whole solution space.

    Avanes~\etal\cite{avanes2008adaptive} present a heuristic for networks in disaster scenarios.
    These networks are a set of DAG-shaped scenarios, out of which one needs to be executed.
    The scenario contains AND- and OR-branches, where AND-branches indicate acitivities that need to be executed in parallel.

    The heuristic first partitions the set into local schedules by using affinity matrices to determine similar
    activities and group them together.
    Then they physically allocate these partitions to groups of disaster responders and tasks within this group.
    They define a constraint system for that.
    The dynamic part deals with changes and distinguishes between retriable and compensation acitivities.
    The heuristic calculates a new execution path with these tasks.
    In general, this heuristic is not as related as it looks, because of very specific workflow and task structure.

    Garg~\etal~\cite{GARG2015256} propose a dynamic scheduling algorithm for heterogeneous grids based on rescheduling.
    The aim is to minimize the makespan, and the experiments were conducted on a single wotkflow with 10 tasks.
    The procesdure involves building a first (static) schedule, priodic resource monitoring and rescheduling the remaining
    tasks.
    The resource model contains resource groups (small tightly-connected sub-clusters), connected between each other.
    For each resuorce group, there is an own scheduler, and an overall global scheduler responsible for distributing
    tasks to groups.
    The authors define the execution time, estimated start time, data ready time,a dn estimated finish time per task.
    The runtimes of tasks depend on processor speeds, are calculated in advance and stored in tables.

    The algorithm first computes bottom levels for all tasks (execution time is average of all possible execution times).
    THe bottom level represents the priority of the task, and tasks are sorted according to these priorities.
    They then go through tasks and map than to such processors that minimize the earliest start times of this task's
    successors.
    To do this, the authors calculate the earliest finishing time of the task across all ressources, along with the
    average communication and computation costs fir the dependent tasks.

    The rescheduling is being triggered when either a load on a resource increases over a threshold, or if a new resource
    is added.
    The algorithm produces a new mapping from scrath, and this mapping is being accepted if the resulting maespan is
    smaller than the previously predicted one.


    \section{Proposal of a new heuristic with slightly refined model}

    The idea is to get rid of the constraint that a processor handles a {\em block} of tasks,
    but favor processor reuse as is done in HEFT.
    Furthermore, this would allow us to handle variability on the fly, by updating
    the bottom levels if some parameters vary, and computing the schedule
    only for the near future...

    \subsection{Model}

    For a task~$u\in V$, we have a memory usage~$m_u$ and execution time~$w_u$.
    For an edge $(u,v)\in E$, we have a data of size $c_{u,v}$.

    For each processor $P_j$, we have a speed~$s_j$, and also two memory bounds:
    $\MM_j$ the processor memory, and $\MC_j$, the size of the communication buffer.
    We can decide to evict some data from the main memory if we are sending the data
    to another processor; it then stays in the communication buffer until it has been sent.

    We keep track of the current ready time of each processor and each communication
    channel, $\rt_j$ and $\rt_{j,j'}$, for all processors~$(j,j')$. Initially, all the ready times
    are set to~$0$.

    We also keep track of the currently available memory, $availM_j$ and $availC_j$,
    on respectively the processor memory and communication buffer.
    Furthermore, $\PD_j$ is a priority queue with the {\em pending data}
    that are in the memory of size $\MM_j$ but may be evicted to be communicated, if
    more memory is needed on~$p_j$. They are ordered by non-decreasing size.
    They correspond to some $c_{u,v}$'s.

    \subsection{Baseline: original HEFT without memories}

    Original HEFT does not consider memory sizes.
    The solutions it provides can be invalid if it schedules tasks to processors without sufficient memories.
    However, these solutions can be viewed as a ``lower bound'' for an actual solution that considers memory sizes.

    HEFT works in two steps.
    In the first step, it calculates the ranks of tasks by computing their non-increasing bottom levels.
    The bottom level of a task is defined as
    $$bl(u) = w_u + \max_{(u,v)\in E} \{c_{u,v} + bl(v)\}$$
    (the max is 0 if there is no outgoing edge).
    The tasks are sorted by non-decreasing ranks.

    In the second step, the algorithm iterates over the ranks and tries to assign the task to the processor where it
    has the earliest finish time.
    We tentatively assign each task to each processor.
    The task's starting time $st_v$ is dictated by the maximum between $rt_j$, and all communications that
    must be orchestrated from predecessor tasks $u\notin T(p_j)$.
    The starting time is then
    \[ST(v, p_j) = \max{ \{rt_j, \max_{ u \in \Pi(v)}\{ FT(u)+ c_{u,v} / \beta , rt_{proc(u), p_j} + c_{u,v} / \beta  \} \} } \]
    Its finish time on $p_j$ is then
    $FT(v,p_j) = st_v + \frac{w_v}{s_j}$.

    Once we have computed all finish times for task~$v$,
    we keep the minimum $FT(v,p_j)$ and assign task~$v$
    to processor~$p_j$.

    \textit{Assignment to processor}
    When assigning the task, we set the ready time of the processor~$j$ $rt_j$ to the finish time of the task.
    For every predecessor of~$v$ that has been assigned to another processor, we adjust ready times on
    communication buffers $rt_{j', j}$ for every predecessor $u$'s processor $j'$: we increase them by the
    communication time $c( u,v) / \beta$.

    \subsection{Heuristics}
    Like the original HEFT, our heuristic consistst of two steps: first, computing task ranks,
    and second, assigning tasks to processors in the order defined in the first step.
    We consider three variants of HEFT accounting for memory usage, which only
    differ in the order they consider tasks to be scheduled.

    \subsubsection{Step 1: calculate task ranks}

    HEFTM-BL orders tasks by non-increasing bottom levels, where the bottom
    level is defined as
    $$bl(u) = w_u + \max_{(u,v)\in E} \{c_{u,v} + bl(v)\}$$
    (the max is 0 if there is no outgoing edge).

    HEFTM-BLC: from the study of the fork (see below), it seems important
    to also account for the size of the data as input of a task,
    to give more priority at tasks with potential large incoming communications.
    For each task, we compute its modified bottom level:
    $$blc(u) = w_u + \max_{(u,w)\in E} \{c_{u,w} + blc(w)\} + \max_{(v,u)\in E} c_{v,u}   . $$

    \skug{avoid having mixed ranks, when the memory size of the lower task is not taken into account}

    HEFTM-MM orders tasks as dictated by MinMem.

    \subsubsection{Task assignment}

    Then, the idea is to pick the next free task in the given order,
    and greedily assign it to a processor, by trying all possible options
    and keeping the most promising one.

    \medskip
    \noindent{\em Tentative assignment of task~$v$ on $p_j$.}\\
    {\bf Step 1.} First, we need to check that for all predecessors~$u$ of~$v$ that are mapped
    on~$p_j$, the data $c_{u,v}$ is still in the memory of~$p_j$,
    i.e., $c_{u,v}\in PD_j$. Otherwise, the finish time is set to~$+\infty$ (invalid choice).

    \smallskip
    \noindent{\bf Step 2.} Next, we check the memory constraint on~$p_j$, by computing
    \[Res = availM_j - m_v - \sum_{u \in \Pi(v), u\notin T(p_j)}  \{c_{u,v}\}
    - \sum_{w\in Succ(v)}  \{c_{v,w}\}.\]

    $T(p_j)$ is the set of tasks already scheduled on $p_j$; by step 1, their files are
    already in the memory of~$p_j$. However, the files from the
    other predecessor tasks must be loaded in memory before executing task~$v$,
    as well as $m_v$ and the data generated for all successor tasks.
    $Res$ is then checking whether there was enough memory; if it is negative,
    it means that we have exceeded the memory of~$p_j$ with this tentative
    assignment.

    In this case ($Res <0$), we try evicting
    some data from memory so that we have enough memory to execute task~$v$.
    We need to evict at least $Res$ data.
    For now, we propose a greedy approach, evicting the smallest files of $\PD_j$ until $Res$ data has been evicted,
    in order to avoid costly communications.
    \AB{FYI We initially discussed evicting the largest files, but this leads to
    large communications and does not seem efficient after all... Maybe we can think of another
    approach that would take into account both data size and bottom level...}
    While tentatively evicting files, we remove them from the list of pending memories and move them into a list
    of memories pending in the communication buffer.
    We keep track of the available buffer size, too - each time a file gets moved into the pending in buffer, the available buffer size is reduced by its weight.

    If we still do not have enough memory after having tentatively evicted all files from $\PD_j$,
    or if while doing so we exceeded the size of the available buffer,
    we set the finish time to~$+\infty$ (invalid choice).

    \smallskip
    \noindent{\bf Step 3.} We tentatively assign task~$v$ on $p_j$.
    Its starting time $st_v$ is dictated by the maximum between $rt_j$, and all communications that
    must be orchestrated from predecessor tasks $u\notin T(p_j)$.
    The starting time is then
    \[ST(v, p_j) = \max{ \{rt_j, \max_{ u \in \Pi(v), u\notin T(p_j)}\{ FT(u) , rt_{proc(u), p_j}\} + c_{u,v} / \beta \} } \]
    Its finish time on $p_j$ is then
    $FT(v,p_j) = ST(v, p_j) + \frac{w_v}{s_j}$.



    \medskip
    \noindent{\em Assignment of task~$v$.}\\
    Once we have computed all finish times for task~$v$,
    we keep the minimum $FT(v,p_j)$ and assign task~$v$
    to processor~$p_j$.
    In detail, we:
    \begin{itemize}
        \item  Evict the file memories that correspond to edge weights that need to be evicted to free the memory.
        We remove these files from pending memories
        $PD_j$, add them to pending data in the communication buffer, and reduce the available buffer size accordingly.
        \item    Calculate the new $availM_j$ on the processor.
        We subtract the weights of all incoming files from predecessors assigned to the same processor,
        and add the weights of outgoing files generated by the currently assigned task.
        \item  For every predecessor of~$v$ that has been assigned to another processor, we adjust ready times on
        communication buffers $rt_{j', j}$ for the processor~$j'$that the predecessor $u$ has been assigned to: we increase them by the
        communication time $c( u,v) / \beta$.
        We also remove the incoming files from either the pending memories or pending data in buffers of these other
        processors, and increase the available memories or available buffer sizes on these processors.
        \item We compute the correct amount of available memory for $p_j$ (for when the task is done).
        For each predecessor that is mapped to the same processor, we remove the pending memory corresponding to the weight of
        the incoming edge, also freeing the same amount of available memory (increasing $availM_j$).
        For each successor, on the other hand, we add the edge weights to pending memories and reduce $availM_j$ by the corresponding
        amount.
    \end{itemize}

    \subsection{Dynamic Scenario}

    In a workflow execution environment, the scheduling method interacts with the runtime environment, which provides information such as resource estimates.
    This information may include, memory usage, runtime, graph structures, or the status of the underlying infrastructure.
    In order to ensure that the information is up to date, a monitoring system observes the workflow execution and collects metrics for tasks and the underlying infrastructure.
    By incorporating dynamic monitoring values, e.g., the resources a task consumed, the runtime environment can incorporate the data into the prediction model to provide more accurate resource predictions.
    Also the underlying infrastructure can change during the workflow execution.
    Examples are processor failures, node recoveries, or acquisition of new nodes.
    However, also when the hardware of the infrastructure does not change, the set of nodes provided as a scheduling target might change due to release or occupation in shared cluster infrastructures.
    As infrastructure information and resource predictions are dynamically updated and provided to the scheduler during the workflow runtime, the previous schedule becomes invalid and a new one must be calculated.

    For state-of-the-art memory prediction methods, a cold-start median prediction error for heterogeneous infrastructures of approximately 15\% is shown~\cite{}.
    Online prediction methods were able to significantly reduce the error during runtime, with the reduction reaching up to one-third of the cold-start error~\cite{baderDiedrichDynamic2023,witt2019learning}.
%For instance, Nadeen~et~al.\cite{} report an error of 10\%, 11\%, and 15\% while the task prediction errors shows a normal and exponential distribution.
%Bader~et~al.~ report a prediction error between 13\% and 17\% for their method, showing an exponential task error distribution.
% @Svetlana, willst du sowas für deine Experimente? Also die Daten, welche du dann konfigurieren kannst?
    Such a dynamic execution environment necessitates for a dynamic scheduling method where the schedule can be recomputed during the workflow execution.

    \subsection{Retracing the effects of change on an existing schedule}
    After the monitoring system has reported changes, we need to assess their impact on the existing schedule.
    These changes can invalidate the schedule (\eg if there is not enough memory for some tasks to execute anymore),
    they can lead to a later finishing time (\eg if some tasks longer and delay other tasks), or they can have no effect (\eg if new processors
    joined the cluster, but the old schedule did not account for them).
    To assess the impact, we need to retrace the schedule.

    First, we find out if at least one processor that had assigned tasks has exited - this instantly invalidates the
    entire schedule.

    We then iterate over all tasks of the workflow in a topological order - any of the orderings given by rankings BL, BLC or MM
    is a topologial ordering.
    We then repeat steps similar to those we did during tentative assignment in our heuristic, except we do not choose a processor
    anymore, but rather see if the current one still fits.

    For each task $v$, we first assess its current memory constraint $Res$ using Step 2 from our heuristic.
    The factors that affect $Res$ are possible changes in $m_v$, in $c_{u,v}$ from predecessors $u$ or $c_{v,w}$ from successors $w$,
    available memory $availM_j$ on the processor (due to either changed $M_j$ or changed memory requirements from other tasks).
    If originally,$Res$ was positive (no files were evicted from memory into the communication buffer), then it has to stay this way -
    otherwise evicted files can invalidate next tasks.
    If original $Res$ was negative, then we need to make sure that evicted files still fit into the communication buffer.
    If either $Res$ is newly negative, or the communication buffer is not large enough, this invalidates the schedule.
    We update the $availM_j$ and $availMC_j$ according to the new memory constraints.

    Then we can re-calculate the finish time of the task on its processor like in Step 3.
    The factors that affect it are changes in own execution time $w_v$ of the tasks, changed ready time of the processor
    (after delayed previous tasks), and changed communication buffer availability.

    Then, after having updated the processor's values, we move on to the next task.

    \subsection{The fork}
    We look at the behavior of these heuristics on a fork graph,
    where there is a root task~$T_0$, producing $n$ files $f_1, \ldots, f_n$
    to be used by tasks $T_1, \ldots, T_n$ ($f_i = c_{0,i}$).

    Without memory, this problem is NP-complete; this is equivalent
    to 2-partition if the tasks have $w_i=a_i$, and all files are of size~$f_i=0$,
    and with two processors. Half of the tasks must be sent to the processor
    on which $T_0$ is not executed, and the optimal makespan is
    $w_0+\frac{1}{2}\sum_{1\leq i \leq n} w_i$.

    However, with an infinite number of identical processors, it can be
    solved in polynomial time: sort tasks by non-decreasing $f_i+w_i$;
    the $k$ tasks with smallest $f_i+w_i$ are then sent to another processor,
    while the remaining $n-k$ tasks are executed locally (try all values of $k$).

    With heterogeneous processors, it is probably NP-complete again
    because we could ensure that there are only two processors fast enough
    and get back to the 2-partition...

    We also had an example where evicting large files first in step 2
    can lead to arbitrarily bad makespan. Consider a fork with $n=2$,
    $f_1=1$, $w_1=2$, $f_2=100$, $w_2=1$, and memory constraint
    imposes that we free one unit of memory before executing one
    of the tasks\ldots Actually the new version with BLC would start
    considering $T_2$ and be fine in this case\ldots


    \AB{Can we prove that we have (maybe) a 2-approximation,
        at least for the fork? What worst-case can we think of? }

    \subsection{Approximation}
    \hmey{Rough notes:}
    Let's use a fork to see how the algorithm behaves and if it provides some approximation. Our current intuition is that, if the memory constraint is ignored, HEFTM-BLc provides a $2$-approximation (to be proved).


    \section{Conclusion}

    \bibliographystyle{IEEEtran}
    \bibliography{references}

\end{document}

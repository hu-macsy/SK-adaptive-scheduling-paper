\documentclass[conference]{IEEEtran}
\IEEEoverridecommandlockouts
% The preceding line is only needed to identify funding in the first footnote. If that is unneeded, please comment it out.
\usepackage{cite}
\usepackage{amsmath,amssymb,amsfonts}
%\usepackage{algorithmic}
\usepackage{graphicx}
\usepackage{textcomp}
\usepackage{xcolor}
\usepackage{algorithmicx}
%\usepackage[colorlinks=true, allcolors=blue]{hyperref}
\usepackage{algpseudocode}
\usepackage{algorithm}
\usepackage{xspace}
\usepackage{hyperref}
\usepackage{numprint}

\def\BibTeX{{\rm B\kern-.05em{\sc i\kern-.025em b}\kern-.08em
T\kern-.1667em\lower.7ex\hbox{E}\kern-.125emX}}
% user-specified commands
%\newcommand{\R}{\mathbb{R}}
\newcommand{\N}{\mathbb{N}}
%\newcommand{\llminimal}{${\|\cdot\|}_2$-minimal\xspace}
\newcommand{\Pro}[1]{\mathbf{Pr} \left[\,#1\,\right]}
\newcommand{\pro}[1]{\mathbf{Pr} [\,#1\,]}
\newcommand{\Ex}[1]{\mathbb{E} \left[\,#1\,\right]}
\newcommand{\ex}[1]{\mathbb{E} [\,#1\,]}
\newcommand{\hit}{- [\pi]_s (H[v,s]-H[u,s])}

\newcommand{\Oh}{\ensuremath{\mathcal{O}}}


\newcommand{\centre}[1]{z_{#1}}
\newcommand{\centres}{Z}
\newcommand{\degree}{\operatorname{deg}}
\newcommand{\maxdeg}{\operatorname{maxdeg}}
\newcommand{\diam}{\operatorname{diam}}
\newcommand{\res}{\operatorname{res}}
\newcommand{\Res}{\operatorname{Res}}
%\newcommand{\cond}{\operatorname{cond}}
\newcommand{\Cond}{\operatorname{Cond}}
\newcommand{\proxy}{\operatorname{proxy}}
\newcommand{\excond}{\operatorname{ex\_cond}}
\newcommand{\exres}{\operatorname{ex\_res}}
\newcommand{\exalg}{\operatorname{ex\_alg}}
\newcommand{\dist}{\operatorname{dist}}
\newcommand{\cost}{\operatorname{c}}
\newcommand{\ord}{\operatorname{ord}}
\newcommand{\Vor}{\operatorname{Vor}}
\newcommand{\M}{\mathbf{M}}
\newcommand{\LL}{\mathbf{L}}
%\newcommand{\L}{\mathbf{L}}
\newcommand{\RR}{\mathbb{R}}
\newcommand{\f}{\hat{f}}
\newcommand{\e}{\mathbf{e}}
\newcommand{\dhat}{d}
\newcommand{\llminimal}{${\|\cdot\|}_2$-minimal\xspace}
\newcommand{\NP}{$\mathcal{NP}$}
\newcommand{\disjbigcup}{\mathop{\dot{\bigcup}}} 
\newcommand{\disjcup}{\mathop{\dot{\cup}}} 
\newcommand{\bigO}{\mathcal{O}} 
\newcommand{\polylog}{\operatorname{polylog}} 
\newcommand{\dotcup}{\stackrel{.}{\cup}}
\newcommand{\myvec}[1]{\mathbf{#1}}
\newcommand{\vent}[2]{\myvec{#1}[#2]}
\newcommand{\size}[1]{\operatorname{size}(#1)}
\newcommand{\distance}[2]{\operatorname{d_{G_p}}[#1][#2]}
\newcommand{\prefix}[1]{\operatorname{prefix}(#1)}
\newcommand{\clz}[1]{\operatorname{clz}(#1)}
\newcommand*\xor{\oplus}


\DeclareMathOperator{\intraWeight}{\it intraWeight}
\DeclareMathOperator{\interWeight}{\it interWeight}
\DeclareMathOperator{\subtreeVol}{\it subtreeVol}
\DeclareMathOperator{\cutWeight}{\it cutWeight}
\DeclareMathOperator{\conduct}{\it conduct}
\DeclareMathOperator{\inCutSet}{\it inCutSet}

\newcommand{\iec}{\textit{i.\,e.},\xspace}
\newcommand{\ie}{\textit{i.\,e.}\xspace}
\newcommand{\Ie}{\textit{I.\,e.}\xspace}
\newcommand{\egc}{\textit{e.\,g.},\xspace}
\newcommand{\eg}{\textit{e.\,g.}\xspace}
\newcommand{\Eg}{\textit{E.\,g.}\xspace}
\newcommand{\etal}{\textit{et al.}\xspace}
\newcommand{\Wlog}{w.\,l.\,o.\,g.\xspace}
\newcommand{\wrt}{w.\,r.\,t.\xspace}
\newcommand{\cf}{cf.\xspace}
\newcommand{\etc}{etc.\xspace}

\newcommand{\networkit}{\textsc{NetworKit}\xspace}
\newcommand{\mswap}{\textsc{TiMEr}\xspace}
\newcommand{\karma}{\textsc{KarMa}\xspace}
\newcommand{\greedyallc}{\textsc{GreedyAllC}\xspace}
\newcommand{\dibap}{\textsc{DibaP}\xspace}
\newcommand{\pdibap}{\textsc{PDibaP}\xspace}
\newcommand{\bubble}{\textsc{Bubble}\xspace}
\newcommand{\smooth}{\textsc{Smooth}}
\newcommand{\bubfosc}{\textsc{Bubble-FOS/C}\xspace}
\newcommand{\bubfost}{\textsc{Bubble-FOS/T}\xspace}
\newcommand{\metis}{\textsc{METIS}\xspace}
\newcommand{\kmetis}{\textsc{kMeTiS}\xspace}
\newcommand{\parmetis}{\textsc{ParMETIS}\xspace}
\newcommand{\libtopomap}{\textsc{LibTopoMap}\xspace}
\newcommand{\kappart}{\textsc{KaPPa}\xspace}
\newcommand{\kahip}{\textsc{KaHIP}\xspace}
\newcommand{\mapkahip}{\textsc{KaHIP\_map}\xspace}
\newcommand{\parhip}{\textsc{ParHIP}\xspace}
\newcommand{\mpipp}{\textsc{MpiPP}\xspace}
\newcommand{\jostle}{\textsc{Jostle}\xspace}
\newcommand{\zoltan}{\textsc{Zoltan}\xspace}
\newcommand{\parkway}{\textsc{Parkway}\xspace}
\newcommand{\graclus}{\textsc{Graclus}\xspace}
\newcommand{\party}{\textsc{Party}\xspace}
\newcommand{\scotch}{\textsc{Scotch}\xspace}
\newcommand{\ptscotch}{\textsc{PT-Scotch}\xspace}
\newcommand{\tree}{\textsc{Tree-Matching}\xspace}
\newcommand{\topomatch}{\textsc{TopoMatch}\xspace}
\newcommand{\xpulp}{\textsc{xtraPulp}\xspace}
\newcommand{\thrsh}{$\mathtt{thrsh}$\xspace}
\newcommand{\trunccons}{\textsc{TruncCons}\xspace}
\newcommand{\consol}{\texttt{Consolidation}\xspace}
\newcommand{\consols}{\texttt{Consolidations}\xspace}
\newcommand{\asspart}{\texttt{AssignPartition}\xspace}
\newcommand{\assclus}{\texttt{AssignCluster}\xspace}
\newcommand{\asssd}{\texttt{AssignSubdomain}\xspace}
\newcommand{\compcen}{\texttt{ComputeCenters}\xspace}
\newcommand{\initcen}{\textsc{LoadBasedInitialCenters}\xspace}
\newcommand{\NN}{\mbox{\rm I$\!$N}}
\newcommand{\closu}[1]{\overline{#1}}
\newcommand{\djoko}{Djokovi\'{c} relation\xspace}
\newcommand{\djokoRelated}{Djokovi\'{c}\xspace related\xspace}
\newcommand{\subE}[1]{{#1}_{\mathcal{E}}}
\newcommand{\comm}{\operatorname{Co}}

\newcommand{\argmin}{\operatorname{argmin}\xspace}
\newcommand{\argmax}{\operatorname{argmax}\xspace}
\newcommand{\cc}{\operatorname{Coco}\xspace}
\newcommand{\divers}{\operatorname{Div}\xspace}
\newcommand{\ccd}{\operatorname{Coco^+}\xspace}
\newcommand{\dil}{\operatorname{dil}\xspace}
\newcommand{\contract}{\operatorname{contract}\xspace}
\newcommand{\assemble}{\operatorname{assemble}\xspace}
\newcommand{\modulo}{\operatorname{mod}\xspace}

\newcommand{\parent}{\operatorname{parent}\xspace}
\newcommand{\oldParent}{\operatorname{oldParent}\xspace}
\newcommand{\newParent}{\operatorname{newParent}\xspace}
\newcommand{\newParentLabel}{\operatorname{newParentLabel}\xspace}
\newcommand{\prefLabel}{\operatorname{prefLabel}\xspace}

\newcommand{\initial}{\textsc{Identity}\xspace}
\newcommand{\initialM}{\textsc{Identity}_{\textsc{g}} \xspace}
\newcommand{\initialMO}{\textsc{Identity}_{\textsc{n}} \xspace}
\newcommand{\random}{\textsc{Random}\xspace}
\newcommand{\randomM}{\textsc{Random}_{\textsc{g}} \xspace}
\newcommand{\randomMO}{\textsc{Random}_{\textsc{n}} \xspace}
\newcommand{\rcm}{\textsc{RCM}\xspace}
\newcommand{\rcmM}{\textsc{RCM}_{\textsc{g}} \xspace}
\newcommand{\rcmMO}{\textsc{RCM}_{\textsc{n}} \xspace}
\newcommand{\durebi}{\textsc{DRB}\xspace}
\newcommand{\durebiM}{\textsc{DRB}_{\textsc{g}} \xspace}
\newcommand{\durebiMO}{\textsc{DRB}_{\textsc{n}} \xspace}
\newcommand{\greedyall}{\textsc{GreedyAll}\xspace}
\newcommand{\greedyallM}{\textsc{GreedyAll}_{\textsc{g}} \xspace}
\newcommand{\greedyallMO}{\textsc{GreedyAll}_{\textsc{n}} \xspace}
\newcommand{\greedyallcM}{\textsc{GreedyAllC} \xspace}
\newcommand{\greedyallcMO}{\textsc{greedyAllC}_{\textsc{n}} \xspace}
\newcommand{\greedymin}{\textsc{GreedyMin}\xspace}
\newcommand{\greedyminM}{\textsc{GreedyMin}_{\textsc{g}} \xspace}
\newcommand{\greedyminMO}{\textsc{GreedyMin}_{\textsc{n}} \xspace}
\newcommand{\greedyminc}{\textsc{GreedyMinC}\xspace}
\newcommand{\greedymincM}{\textsc{GreedyMinC}_{\textsc{g}} \xspace}
\newcommand{\greedymincMO}{\textsc{GreedyMinC}_{\textsc{n}} \xspace}
\newcommand{\walshawlarge}{\textsc{WalshawLarge}\xspace}
\newcommand{\complexnets}{\textsc{ComplexNets}\xspace}
\newcommand{\gpmetis}{\textsc{gpMetis}\xspace}
\newcommand{\ndmetis}{\textsc{ndMetis}\xspace}

\newcommand{\seacode}{\textsc{SEAmap}\xspace}
\newcommand{\ouralgo}{\textsc{ParHIP\_map}\xspace }


\newcommand{\real}{\textsc{real}\xspace}
\newcommand{\ba}{\textsc{BA}\xspace}
\newcommand{\rhg}{\textsc{RHG}\xspace}
\newcommand{\rmat}{\textsc{Rmat}\xspace}


%% \newcommand{\cone}{\texttt{cSc}\xspace}
%% \newcommand{\ctwo}{\texttt{cId}\xspace}
%% \newcommand{\cthree}{\texttt{cGr}\xspace}
%% \newcommand{\cfour}{\texttt{cLb}\xspace}
\newcommand{\cone}{\mathtt{c1}\xspace}
\newcommand{\ctwo}{\mathtt{c2}\xspace}
\newcommand{\cthree}{\mathtt{c3}\xspace}
\newcommand{\cfour}{\mathtt{c4}\xspace}



\newcommand{\algo}[1]{\textsc{#1}}
\newcommand{\bottomlevel}[1]{\underline{l}_{#1}} % underline short italic
\newcommand{\criticalpath}{\mathcal{P}}
\newcommand{\parents}[1]{\,\Pi_{#1}}
\newcommand{\children}[1]{\,C_{#1}}
\newcommand{\cluster}{\,\mathcal{S}}

\newcommand{\heftmm}{\algo{HEFTM\_MM}\xspace}
\newcommand{\heftbl}{\algo{HEFTM\_BL}\xspace}
\newcommand{\heftblc}{\algo{HEFTM\_BLC}\xspace}


\newcommand{\MM}{M}
\newcommand{\MC}{MC}
\newcommand{\rt}{rt}
\newcommand{\curM}{curM}
\newcommand{\curC}{curC}
\newcommand{\PD}{PD}

\newcommand{\skug}[1]{{\color{blue}[SK: #1]}}
\newcommand{\hmey}[1]{{\color{red}[HM: #1]}}
\newcommand{\AB}[1]{{\color{purple}[AB: #1]}}

\begin{document}

    \title{Adaptive Scheduling of Scientific Workflows\\
%{\footnotesize \textsuperscript{*}Note: Sub-titles are not captured in Xplore and
%should not be used}
    \thanks{Identify applicable funding agency here. If none, delete this.}
    }

%\author{\IEEEauthorblockN{1\textsuperscript{st} Given Name Surname}
%\IEEEauthorblockA{\textit{dept. name of organization (of Aff.)} \\
%\textit{name of organization (of Aff.)}\\
%City, Country \\
%email address or ORCID}
%\and
%\IEEEauthorblockN{2\textsuperscript{nd} Given Name Surname}
%\IEEEauthorblockA{\textit{dept. name of organization (of Aff.)} \\
%\textit{name of organization (of Aff.)}\\
%City, Country \\
%email address or ORCID} }


    \maketitle

    \begin{abstract}
        TODO: Insert abstract
    \end{abstract}

    \begin{IEEEkeywords}
        Scheduling, Adaptive Scheduling, DAG
    \end{IEEEkeywords}

    \section{Introduction}

    %%% CONTEXT %%%
    Only mapping tasks to processors is not enough.
    Reusing processors is important to achieve good makespans.

    %%% MOTIVATION %%%
    HEFT is one of the most popular heuristics (cite a review).
    However, existing HEFT variations do not take memory sizes into consideration (\skug{check in related work if true!}).

    %%% CONTRIBUTION %%%
    In this paper, we formulate the problem that takes memories into consideration.
    We propose three HEFT-based heuristics that take memory size into account: \heftbl, \heftblc, and \heftmm.
    The difference is the way they order tasks for scheduling.
    We compare our heuristics to a baseline HEFT variant that does not take memory sizes into account.

    We find that our heuristics are able to schedule all workflows correctly, and produce makespans similar to the baseline.


    \section{Model}

    \skug{TODO:unify the notation with the table!!}

    For a task~$u\in V$, we have a memory usage~$m_u$ and execution time~$w_u$.
    For an edge $(u,v)\in E$, we have a data of size $c_{u,v}$.

    For each processor $P_j$, we have a speed~$s_j$, and also two memory bounds:
    $\MM_j$ the processor memory, and $\MC_j$, the size of the communication buffer.
    We can decide to evict some data from the main memory if we are sending the data
    to another processor; it then stays in the communication buffer until it has been sent.

    We keep track of the current ready time of each processor and each communication
    channel, $\rt_j$ and $\rt_{j,j'}$, for all processors~$(j,j')$. Initially, all the ready times
    are set to~$0$.

    We also keep track of the currently available memory, $availM_j$ and $availC_j$,
    on respectively the processor memory and communication buffer.
    Furthermore, $\PD_j$ is a priority queue with the {\em pending data}
    that are in the memory of size $\MM_j$ but may be evicted to be communicated, if
    more memory is needed on~$p_j$. They are ordered by non-decreasing size.
    They correspond to some $c_{u,v}$'s.


    \begin{table}[h]
        \begin{center}
            \begin{tabular}{rl}
                \hline
                \textbf{Symbol}                       & \textbf{Meaning}                                         \\
                \hline
                $G = (V, E)$                          & Workflow graph, set of vertices (tasks) and edges        \\
                $\parents{u}$, $\children{u}$         & Parents of a task $u$, children of a task $u$            \\
                $m_u$                                 & Memory weight of task $u$                                \\
                $w_u$                                 & Workload of task $u$  (normalized execution time)          \\
                $c_{u,v}$                             & Communication volume along the edge $(u,v)\in E$         \\
                $F$, $\mathcal{F}$                    & A partitioning function and the partition it creates     \\
                $V_i$                                 & Block number $i$                                         \\ %\wrt~some $F$   \\
                $\cluster$, $k$                    & Computing system and its number of processors           \\
                $p_j$, proc($V_i$)                          & Processor number $j$, processor of block $V_i$                 \\
                $M_j$, $s_j$                               & Memory size and speed of processor $p_j$                          \\
                $\beta$                     & Bandwidth in the compute system                                \\
                $\bottomlevel{u}$                      & Bottom weight of task $u$ \\
                $\mu_G$, $\mu_i$ & Makespan of the entire workflow $G$ and of a block $V_i$               \\
                $\Gamma = (\mathcal{V}, \mathcal{E})$                      & Quotient graph, its vertices and its edges        \\
                $r_u$, $r_{V_i}$                            & Memory requirement of task $u$ and of block $V_i$                 \\
                $r_{\max}$                       & Maximum memory requirement in a workflow                              \\
                $\criticalpath$                       & Critical path in a workflow                              \\
                \hline
            \end{tabular}
        \end{center}
        \caption{Notation.} \label{tabnotation}
    \end{table}

    \section{Related work: new}

    \paragraph{HEFT-based algorithms}

    Introduced in 2002, HEFT~\cite{topcuoglu2002performance} has become a base for multiple heuristics in workflow (~\cite{SHI2006665, SANDOKJI2019482}) scheduling
    and in cloud-oriented environments(\cite{samadi2018eheft}).
    HEFT is even combined with reinforcement learning techniques(\cite{yano2022cqga}).

    HEFT-based or list-based heuristics are more applicable to very large workflows in comparison to greedy
    or exact algorithms(~\cite{de2012provenance},~\cite{rahman2013}).


    \paragraph{Memory-aware scheduling algorithms}
    Incorporating processor memory sizes into the scheduling problem is another layer of complexity in a scheduling problem.

    Different models of memory available on processors and memory requirements of tasks.
    The most straightforward memory representations assume no memory requirement on tasks of the workflow and single
    memories on each processor(\eg~\cite{rodriguez2019exploration}).
    In an assumed dual-memory systems(\cite{herrmann2014memory}) a processor can have access to a memory of two different kinds,
    and each task can be executed on one sort of memory.
    Cloud-oriented model can include costs associated with memory usage(\cite{liang2020memory}).


    The algorithm presented by Yao\etal~\cite{yao2022memory} looks similar to our heuristics on the first glance, but the
    model the authors adopt differs from the one assumed in this paper.
    The biggest difference is in the way processor memory is represented.
    Each processor has an own internal memory and all processors share a common external one.
    The internal (local) memory is used to store the task files.
    The external memory is used to store evicted files to make room for the execution of a task on a processor.
    However, this marks the decisive difference to our approach: once the files have been evicted into the communication
    buffer in our approach, they can only be transferred to another processor.
    In~\cite{yao2022memory} however, the same processor can evict files and then take them back.
    This makes the approach not directly comparable with ours.
    Other differences in the model include edge weights in the workflow graph.
    Each edges in~\cite{yao2022memory} has two weights - the size of the files transferred along it,
    and the time of communication along this edge, while we assume a singular weight that represents file size and also
    determines the transfer time (along with the bandwidth).
    The tasks themselves have no memory requirements, but need to hold all their incoming and outgoing files.
    They assume that the {\em minimum} local memory should be bigger than the memory requirement of the largest task.


    \section{Proposal of a new heuristic with slightly refined model}

    The idea is to get rid of the constraint that a processor handles a {\em block} of tasks,
    but favor processor reuse as is done in HEFT.
    Furthermore, this would allow us to handle variability on the fly, by updating
    the bottom levels if some parameters vary, and computing the schedule
    only for the near future...

    \subsection{Baseline: original HEFT without memories}

    Original HEFT does not consider memory sizes.
    The solutions it provides can be invalid if it schedules tasks to processors without sufficient memories.
    However, these solutions can be viewed as a ``lower bound'' for an actual solution that considers memory sizes.

    HEFT works in two steps.
    In the first step, it calculates the ranks of tasks by computing their non-increasing bottom levels.
    The bottom level of a task is defined as
    $$bl(u) = w_u + \max_{(u,v)\in E} \{c_{u,v} + bl(v)\}$$
    (the max is 0 if there is no outgoing edge).
    The tasks are sorted by non-decreasing ranks.

    In the second step, the algorithm iterates over the ranks and tries to assign the task to the processor where it
    has the earliest finish time.
    We tentatively assign each task to each processor.
    The task's starting time $st_v$ is dictated by the maximum between $rt_j$, and all communications that
    must be orchestrated from predecessor tasks $u\notin T(p_j)$.
    The starting time is then
    \[ST(v, p_j) = \max{ \{rt_j, \max_{ u \in \Pi(v)}\{ FT(u)+ c_{u,v} / \beta , rt_{proc(u), p_j} + c_{u,v} / \beta  \} \} } \]
    Its finish time on $p_j$ is then
    $FT(v,p_j) = st_v + \frac{w_v}{s_j}$.

    Once we have computed all finish times for task~$v$,
    we keep the minimum $FT(v,p_j)$ and assign task~$v$
    to processor~$p_j$.

    \textit{Assignment to processor}
    When assigning the task, we set the ready time of the processor~$j$ $rt_j$ to the finish time of the task.
    For every predecessor of~$v$ that has been assigned to another processor, we adjust ready times on
    communication buffers $rt_{j', j}$ for every predecessor $u$'s processor $j'$: we increase them by the
    communication time $c( u,v) / \beta$.

    \subsection{Heuristics}
    Like the original HEFT, our heuristic consistst of two steps: first, computing task ranks,
    and second, assigning tasks to processors in the order defined in the first step.
    We consider three variants of HEFT accounting for memory usage, which only
    differ in the order they consider tasks to be scheduled.

    \subsubsection{Step 1: calculate task ranks}

    HEFTM-BL orders tasks by non-increasing bottom levels, where the bottom
    level is defined as
    $$bl(u) = w_u + \max_{(u,v)\in E} \{c_{u,v} + bl(v)\}$$
    (the max is 0 if there is no outgoing edge).

    HEFTM-BLC: from the study of the fork (see below), it seems important
    to also account for the size of the data as input of a task,
    to give more priority at tasks with potential large incoming communications.
    For each task, we compute its modified bottom level:
    $$blc(u) = w_u + \max_{(u,w)\in E} \{c_{u,w} + blc(w)\} + \max_{(v,u)\in E} c_{v,u}   . $$

    \skug{avoid having mixed ranks, when the memory size of the lower task is not taken into account}

    HEFTM-MM orders tasks as dictated by MinMem.

    \subsubsection{Task assignment}

    Then, the idea is to pick the next free task in the given order,
    and greedily assign it to a processor, by trying all possible options
    and keeping the most promising one.

    \medskip
    \noindent{\em Tentative assignment of task~$v$ on $p_j$.}\\
    {\bf Step 1.} First, we need to check that for all predecessors~$u$ of~$v$ that are mapped
    on~$p_j$, the data $c_{u,v}$ is still in the memory of~$p_j$,
    i.e., $c_{u,v}\in PD_j$. Otherwise, the finish time is set to~$+\infty$ (invalid choice).

    \smallskip
    \noindent{\bf Step 2.} Next, we check the memory constraint on~$p_j$, by computing
    \[Res = availM_j - m_v - \sum_{u \in \Pi(v), u\notin T(p_j)}  \{c_{u,v}\}
    - \sum_{w\in Succ(v)}  \{c_{v,w}\}.\]

    $T(p_j)$ is the set of tasks already scheduled on $p_j$; by step 1, their files are
    already in the memory of~$p_j$. However, the files from the
    other predecessor tasks must be loaded in memory before executing task~$v$,
    as well as $m_v$ and the data generated for all successor tasks.
    $Res$ is then checking whether there was enough memory; if it is negative,
    it means that we have exceeded the memory of~$p_j$ with this tentative
    assignment.

    In this case ($Res <0$), we try evicting
    some data from memory so that we have enough memory to execute task~$v$.
    We need to evict at least $Res$ data.
    For now, we propose a greedy approach, evicting the smallest files of $\PD_j$ until $Res$ data has been evicted,
    in order to avoid costly communications.
    \AB{FYI We initially discussed evicting the largest files, but this leads to
    large communications and does not seem efficient after all... Maybe we can think of another
    approach that would take into account both data size and bottom level...}
    While tentatively evicting files, we remove them from the list of pending memories and move them into a list
    of memories pending in the communication buffer.
    We keep track of the available buffer size, too - each time a file gets moved into the pending in buffer, the available buffer size is reduced by its weight.

    If we still do not have enough memory after having tentatively evicted all files from $\PD_j$,
    or if while doing so we exceeded the size of the available buffer,
    we set the finish time to~$+\infty$ (invalid choice).

    \smallskip
    \noindent{\bf Step 3.} We tentatively assign task~$v$ on $p_j$.
    Its starting time $st_v$ is dictated by the maximum between $rt_j$, and all communications that
    must be orchestrated from predecessor tasks $u\notin T(p_j)$.
    The starting time is then
    \[ST(v, p_j) = \max{ \{rt_j, \max_{ u \in \Pi(v), u\notin T(p_j)}\{ FT(u) , rt_{proc(u), p_j}\} + c_{u,v} / \beta \} } \]
    Its finish time on $p_j$ is then
    $FT(v,p_j) = ST(v, p_j) + \frac{w_v}{s_j}$.



    \medskip
    \noindent{\em Assignment of task~$v$.}\\
    Once we have computed all finish times for task~$v$,
    we keep the minimum $FT(v,p_j)$ and assign task~$v$
    to processor~$p_j$.
    In detail, we:
    \begin{itemize}
        \item  Evict the file memories that correspond to edge weights that need to be evicted to free the memory.
        We remove these files from pending memories
        $PD_j$, add them to pending data in the communication buffer, and reduce the available buffer size accordingly.
        \item    Calculate the new $availM_j$ on the processor.
        We subtract the weights of all incoming files from predecessors assigned to the same processor,
        and add the weights of outgoing files generated by the currently assigned task.
        \item  For every predecessor of~$v$ that has been assigned to another processor, we adjust ready times on
        communication buffers $rt_{j', j}$ for the processor~$j'$that the predecessor $u$ has been assigned to: we increase them by the
        communication time $c( u,v) / \beta$.
        We also remove the incoming files from either the pending memories or pending data in buffers of these other
        processors, and increase the available memories or available buffer sizes on these processors.
        \item We compute the correct amount of available memory for $p_j$ (for when the task is done).
        For each predecessor that is mapped to the same processor, we remove the pending memory corresponding to the weight of
        the incoming edge, also freeing the same amount of available memory (increasing $availM_j$).
        For each successor, on the other hand, we add the edge weights to pending memories and reduce $availM_j$ by the corresponding
        amount.
    \end{itemize}

    \subsection{The fork}
    We look at the behavior of these heuristics on a fork graph,
    where there is a root task~$T_0$, producing $n$ files $f_1, \ldots, f_n$
    to be used by tasks $T_1, \ldots, T_n$ ($f_i = c_{0,i}$).

    Without memory, this problem is NP-complete; this is equivalent
    to 2-partition if the tasks have $w_i=a_i$, and all files are of size~$f_i=0$,
    and with two processors. Half of the tasks must be sent to the processor
    on which $T_0$ is not executed, and the optimal makespan is
    $w_0+\frac{1}{2}\sum_{1\leq i \leq n} w_i$.

    However, with an infinite number of identical processors, it can be
    solved in polynomial time: sort tasks by non-decreasing $f_i+w_i$;
    the $k$ tasks with smallest $f_i+w_i$ are then sent to another processor,
    while the remaining $n-k$ tasks are executed locally (try all values of $k$).

    With heterogeneous processors, it is probably NP-complete again
    because we could ensure that there are only two processors fast enough
    and get back to the 2-partition...

    We also had an example where evicting large files first in step 2
    can lead to arbitrarily bad makespan. Consider a fork with $n=2$,
    $f_1=1$, $w_1=2$, $f_2=100$, $w_2=1$, and memory constraint
    imposes that we free one unit of memory before executing one
    of the tasks\ldots Actually the new version with BLC would start
    considering $T_2$ and be fine in this case\ldots


    \AB{Can we prove that we have (maybe) a 2-approximation,
        at least for the fork? What worst-case can we think of? }

    \subsection{Approximation}
    \hmey{Rough notes:}
    Let's use a fork to see how the algorithm behaves and if it provides some approximation. Our current intuition is that, if the memory constraint is ignored, HEFTM-BLc provides a $2$-approximation (to be proved).

    \section{Experimental evaluation}
    \subsection{Setup}
    \label{sec:setup}

    All algorithms are implemented in C++ and compiled with g++ (v.11.2.0).
    The experiments are executed on workstations with 192 GB RAM and 2x 12-Core Intel Xeon 6126 @3.2 GHz
    and CentOS 8 as OS.
    Code, input data, and experiment scripts are available for review under \url{TODO:insert link}.

    Next, we describe the set of workflows used in the evaluation and then the clusters on which the
    workflows are scheduled.

    \subsubsection{Workflow instances}
    The input data for the experiments consists of two sets of workflows: real-world workflows
    obtained from~\cite{ewels2020nf} and workflows obtained by simulating real-world workflows
    with the WFGen generator~\cite{COLEMAN202216}.
    First, we discuss how the graph topology is generated,
    then we focus on the weights associated to tasks and edges.

    \paragraph{Workflow graphs}
    For real-world workflows, their nextflow definition (see~\cite{ewels2020nf}) was downloaded from the
    respective repository and transformed into .dot format using the nextflow option ``-with-dag''.
    The resulting DAG contains many pseudo-tasks that are only internal representations in nextflow
    (and not actual tasks); that is why we removed them.

    For the simulated workflows, the graph is produced by the WFGen generator, based on a {\em model workflow} and
    the desired number of tasks.
    The model workflows are described on the WFGen website, and we used the following ones:
    1000Genome, BLAST, BWA, Epigenomics, Montage, Seismology, and SoyKB.
    Other models could not be generated without errors.
%
    As number of tasks, we use: 200, \numprint{1000}, \numprint{2000}, \numprint{4000}, \numprint{8000}, \numprint{10000},
    \numprint{15000}, \numprint{18000}, \numprint{20000}, \numprint{25000}, \numprint{30000}.
    We divide the workflows into three groups by size: small ones with up to \numprint{8000} tasks, middle
    ones with \numprint{10000} to \numprint{18000} tasks, and big ones with \numprint{20000} to \numprint{30000} tasks.
    Note that for some workflow models, such as SoyKB or Montage, only a subset of the workflow sizes could be generated,
    because the workflow generator either took a disproportionally long time or yielded errors.

    Overall, this yields four workflow types, denoted by real, small, mid, and big.

    \paragraph{Generation of task and edge weights}
    For the real-world workflows, we use historical data files provided by Bader~\etal~\cite{lotaru}.
    The columns in these files are measured Linux PS stats, acquired during an execution of a nextflow workflow.
    Each row corresponds to an execution of one task on one cluster node.
    Since the operating system cannot distinguish between (a) the RAM the task uses for itself and (b) the RAM it uses
    to store files that were sent or received from other tasks, the values in the historical data are total memory requirements (input/output files plus memory consumption of the computation).
    In a similar manner, the historical data provided by~\cite{lotaru} do not store the actual weights of edges between tasks, but only the overall
    size of files that the task sends to all its children.
    For each task, historical data can contain multiple values, obtained from the runs with different input sizes.
    To avoid underestimation, we take the maximum among all values from different runs on the same cluster node.
    Not all tasks have historical runtime data stored in the tables.
    In fact, for two workflows, Bader~\etal do not provide data for more than 50\% of the tasks.
    For two more, around 40\% of tasks have no historical runtime data stored.
    Hence, in the absence of historical data about a task, we give it a weight of~1.
%
    Because the historical data contains absolute measured values and the cluster node information is a relative value,
    we normalize all values extracted from the historical data by the smallest one.
    This way, task memory weights fit into the memory of the cluster nodes.
    Additionally, this way, the tasks without historical data receive less insignificant values compared to tasks with historical data.

    For the simulated workflows, we generate random execution times and memory weights for each task as well as
    edge weights for each precedence constraint.
    We generate uniformly distributed values between $1$ and $10$ for edge weights,
    $1$ and $1000$ for the workloads,
    and $1$ and $192$ for memory weights.
    When doing so, we try to mimic the weights observed in the historical data, hence \eg the low lower bounds for the
    workloads.


    \subsubsection{Target computing systems}
    To fully benefit from the historical data, the  {\em default} experimental environment
    that we consider is a cluster based on the same six
    kinds of real-world machines that were used in the experimental evaluation in~\cite{lotaru}.
    We set the number of each kind of node to six, thus having 36 processors in total. % the whole cluster.

        Each machine has a (normalized) CPU speed and a memory size (in GB), and we list them as (name, speed, memory):
        ($local$, 4, 16) -- very slow machines; ($A1$, 32, 32), ($A2$, 6, 64), ($N1$, 12, 16) -- average machines; ($N2$, 8, 8) -- machine with very small memory; and ($C2$, 32, 192) -- {\em luxury} machine with high speed and large memory
        (see Table~\ref{tab:procs} in Appendix~\ref{sub:appx-cluster-config}~\cite{daghetpart_full_version}).
    Note that, by nature,
    memory sizes are normalized values, relative to each other, too.
    Therefore, we additionally normalize memory weights of real-world workflows
    to the maximum size of 192 to make sure they fit.
    For simulated workflows, we increase memory sizes proportionally until the task
    with the biggest memory requirement still
    has a processor it could be executed on.

    To test various settings, we also vary the cluster configuration and consider
    variants of the default cluster presented above:\\
    $\bullet$ {\em Small} and {\em large} clusters. While the default cluster consists of 36 nodes (6 of each kind), we
    consider a small cluster with one processor of each kind ($6$ processors in total), and a
    large cluster with ten processors of each kind ($60$ processors in total). \skug{leaving the big cluster inside for
    the case we do want to include it}\\
    $\bullet$ {\em Fat-and-thin} cluster. In this cluster, the fastest processors have the smallest amount of memory (thin),
    while the slowest processors have the largest memories (fat).
    The overall amount of memory in the cluster, as well as the overall speed of all processors remain the same as in the
            {\em normal} cluster.
%\end{itemize}


    \subsection{Results}
    \subsubsection{Normal cluster}

    On the normal cluster, the baseline produces invalid results in $40$\% of small workflows and $66.6$\% of middle-sized
    and big workflows.
    In most cases, the failures of the baseline can be traced to choosing the processor with not enough memory when several
    options are available.
    Due to not taking memories into account, the baseline chooses a processor with already full memory, while there can
    be another processor with the same speed that still has enough memory.
    In this case, \heftbl produces the same makespan while producing a valid solution.
    The heuristics produce valid results by default and also manage to find a solution in all cases (on all workflow sizes).

    On the normal cluster and among small workflows, the makespans produced by \heftbl and \heftblc are $1$\% worse than the baseline,
    \heftmm $19$\% worse.
    For middle-sized ones \heftmm is $12.3$\% worse, the rest is also $1-2$\% worse.
    For the big workflows however, the results of \heftbl and \heftblc are $<1$\% worse and \heftmm is only $9$\% worse.
    The heuristics pay off more on larger workflows.
    \subsubsection{Memory usage}
    \subsubsection{Impact of cluster size}
    On a cluster with only 6 processors (1 of each kind), the baseline makes more failures: from $43.4$\%  (small workflows)
    to $88.9$\% (mid-sized) and $94.4$\% (big workflows).
    These values also become more unattainable: the makespans heuristics produce are worse by $3-12$\% (\heftbl) to $12-52$\%
    (\heftmm).
    However, these comparisons are less expressive, because most of baseline solutions are invalid.
    \skug{TODO: add numbers about failures on the small cluster with the fixed bug}

    \subsubsection{Impact of cluster composition}
    On the {\em fat-and-thin} cluster, the baseline produces $60$\% invalid results on small workflows, $83.3$\% on
    middle-sized and $88.9$\% on big workflows, and therefore performs worse than on the normal cluster.

    The relative makespans are $14-29$\% worse for small workflows, $22-40$\% worse on middle-sized workflows and $16-35$\%
    on big workflows.
    \wrt failures, all heuristics again manage to schedule all workflows.
    \subsubsection{Behavior of workflow types when scaling workflow size}
    \subsubsection{Running times}
    The geometric means of runtimes of all heuristics on small workflows lie between $0.9$ and $2.2$ seconds.
    For middle-sized workflows, the runtimes for \heftbl and \heftblc are close to $2$ minutes, while \heftmm
    takes $5.1$ minutes.
    For the big workflows, again, \heftbl and \heftblc show a better runtime ($5$ and $5.4$ minutes, respectively).
    \heftmm takes $13.9$ minutes on average.
    \subsubsection{Summary}

    \section{Conclusion}

    \bibliographystyle{IEEEtran}
    \bibliography{references}

\end{document}
